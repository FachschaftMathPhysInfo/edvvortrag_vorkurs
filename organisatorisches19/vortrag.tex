%pdf-latex

%header nur vom MÜSLI-Vortrag geklaut
% !TEX root = present.tex
% !TEX encoding = UTF-8 Unicode

\documentclass[
    t,
    ngerman,
    hyperref={
        colorlinks=false
    }
    ]{beamer} %compress,

%% Pakete laden...
  \usepackage[T1]{fontenc}
  \usepackage[utf8]{inputenc}
  \usepackage{
      babel,
      bookmark,
      booktabs,
%      blindtext,
      colortbl,
%      eurosym,
      graphicx,
      hyperref,
%      libertine,
      microtype,
      pifont,
      pgfpages,
      tikz,
      wrapfig,
      qrcode,
%      xspace,
  }


%% Design festlegen...
  \mode<presentation>{
%      \useoutertheme[subsection=false]{smoothbars}
      \useinnertheme{rectangles} % rectangles, circles, rounded
      \usecolortheme[RGB={153,0,0}]{structure}
      \definecolor{unihd}{RGB}{153,0,0}
      \definecolor{dark}{RGB}{115,0,0}
      \definecolor{light}{RGB}{241,229,229}
      \usecolortheme{whale}
         \usecolortheme{orchid}
%      \usecolortheme{beaver}
%      \setbeamercovered{transparent}
      \beamertemplatenavigationsymbolsempty
%      \setbeameroption{show notes on second screen}
      \setbeamertemplate{note page}[plain]
      \logo{\includegraphics[width=3.5cm]{./media/fs-logo-small.pdf}}

  }

%% nützliche Definitionen...
  \graphicspath{{media/}}

%% Titelinformationen...
  \title[EDV]{Serviceangebote Studium\\(VPN, eduroam, URZ, drucken, UB)}
  \author[
    Patrick und Chris
  ]{
    Patrick Dammann\\{\scriptsize\url{patrick@mathphys.stura.uni-heidelberg.de}}\\
    Christian Heusel\\{\scriptsize\url{chris@mathphys.stura.uni-heidelberg.de}}
}
%  \institute{  \includegraphics[width=5.5cm]{fs-logo-big} }

  \date[02.10.2017]{\vspace*{-2em}\\ 02. Oktober 2018}

  \hypersetup{
      pdfauthor={Patrick Damann, Christian Heusel},
      pdftitle={Serviceangebote Studium},
      pdfsubject={Serviceangebote Studium},
      pdfkeywords={1},
      pdfpagelayout={SinglePage},
  }



\newenvironment{rcases}{%
  \left.\renewcommand*\lbrace.%
  \begin{cases}}%
{\end{cases}\right\rbrace}


\begin{document}

\begin{frame}[fragile]
    \maketitle{}
\end{frame}

\begin{frame}{Inhalt des Vortrags}
    \begin{minipage}[t]{0.515\textwidth}
        \tableofcontents[hideallsubsections, sections={1-5}]
    \end{minipage}
    \begin{minipage}[t]{0.475\textwidth}
        \tableofcontents[hideallsubsections, sections={6-11}]
    \end{minipage}
\end{frame}

\begin{frame}{Wer will mir hier was erzählen!?}
    \vfill
    \begin{center}
        {\Large Janina Rastetter} \\
        \mail{jrastetter@mathi.uni-heidelberg.de} \\
        \vspace{1em}
        {\Large Christian Heusel } \\
        \mail{chris@mathphys.stura.uni-heidelberg.de} \\
    \end{center}
\end{frame}

\begin{frame}{Präambel}
    \Large
    \begin{center}
        \only<1->{%
            {\huge \DejaSans{} 😇} Es wird die Folien zum Download geben! {\huge \DejaSans{} 😇}\\[0.7em]
        }
        \only<1>{\includegraphics[scale=0.6]{images/xkcd_1.png}}
        \only<2->{%
            {\huge \DejaSans{} 😺} Stellt Fragen, im Währenden oder am Ende! {\huge \DejaSans{} 😺}\\[0.7em]
        }
        \only<3->{%
            {\huge \DejaSans{} 😱} Es wird viele Informationen geben! {\huge \DejaSans{} 😱}\\[0.7em]
        }
        \only<4->{%
            \includegraphics[]{images/lets_go.png}
        }
    \end{center}
\end{frame}

%%%%%%%%%%%%%%%%%%%%%%%%%%%%%%%%%%%%%%%%%%%%%%%%%%%%%%%%%%%%
% Digitale Infrastruktur
%%%%%%%%%%%%%%%%%%%%%%%%%%%%%%%%%%%%%%%%%%%%%%%%%%%%%%%%%%%%

\section{Digitale Infrastruktur}
\begin{frame}{Übersicht: Digitale Infrastruktur}
    \large
    \begin{itemize}
        \item{WLAN}
        \item{Drucken}
        \item{VPN}
        \item{CIP-Pool}
        \item{Uni-Mail-Adresse}
    \end{itemize}
\end{frame}

%%%%%%%%%%%%%%%%%%%%%%%%%%%%%%%%%%%%%%%%%%%%%%%%%%%%%%%%%%%%
% WLAN
%%%%%%%%%%%%%%%%%%%%%%%%%%%%%%%%%%%%%%%%%%%%%%%%%%%%%%%%%%%%

\subsection{WLAN}
\begin{frame}{Eduroam}
    \begin{minipage}[t]{0.715\textwidth}
        \vspace{-2em}
        \large \textbf{Was ist das Eduroam?}
        \normalsize
        \begin{itemize}
            \item<1-> große Abdeckung in Heidelberg
            \item<2-> Zugang in vielen weiteren Ländern der Welt
            \item<3-> Verschlüsselter Internetzugang
            \item<4-> für den täglichen Gebrauch
        \end{itemize}
        \only<1>{
            \begin{center}
                \vspace{-6.5em}
                \includegraphics[width=1.3\textwidth]{eduroam_map.png}
            \end{center}
        }
    \end{minipage}
    \begin{minipage}[t]{0.25\textwidth}
        \begin{flushright}
            \includegraphics[width=0.6\textwidth]{eduroam450pix.png}
        \end{flushright}
    \end{minipage}
\end{frame}

\begin{frame}{Eduroam einrichten -- leicht gemacht}
    \large \url{https://cat.eduroam.org} \\
    \begin{minipage}[t]{0.515\textwidth}
        \begin{enumerate}
            \item<+-> Link aufrufen
            \item<+-> \glqq{}Uni Heidelberg\grqq{} suchen
            \item<+-> Tool herunterladen
            \item<+-> Tool ausführen
            \item<+-> Zugangsdaten eingeben
            \item<+-> Fertig!
        \end{enumerate}
    \end{minipage}
    \begin{minipage}[t]{0.4\textwidth}
        \vspace{0cm}
        \begin{center}
            \qrcode[height=0.8\textwidth, link]{https://cat.eduroam.org}
        \end{center}
    \end{minipage}
\end{frame}


\begin{frame}{Übersicht: Verfügbare WiFi-Netze}
    \large \textbf{Welche Netze gibt es noch?}
    \normalsize
    \begin{itemize}[<+->]
        \item \texttt{eduroam} \\
            für den täglichen Gebrauch
        \item \texttt{UNI-HEIDELBERG} \\
            Internet nur über VPN, kaum Ports gesperrt
        \item {\color{gray}(\texttt{Heidelberg4You})} \\
            Unsicher, unzuverlässig, funktioniert aber einfach so
        \item {\color{gray}(\texttt{UNI-WEBACCESS})} \\
            Unsicher, für Tagungsgäste
    \end{itemize}
\end{frame}

%%%%%%%%%%%%%%%%%%%%%%%%%%%%%%%%%%%%%%%%%%%%%%%%%%%%%%%%%%%%
% Drucken
%%%%%%%%%%%%%%%%%%%%%%%%%%%%%%%%%%%%%%%%%%%%%%%%%%%%%%%%%%%%

\subsection{Drucken}
\begin{frame}{Drucken}
    \large \textbf{Wie drucke ich denn in der Uni?}
    \normalsize
    \begin{enumerate}
        \item die zu druckende Datei im Web-Portal hochladen \\
            {\footnotesize
                \begin{itemize}
                    \item \url{https://ricoh-eop.urz.uni-heidelberg.de} \\
                    \item \url{https://www.urz.uni-heidelberg.de/de/mobiles-drucken}
                \end{itemize}
            }
        \item zu einem der FollowMe-Printer laufen
        \item Studiausweis einstecken
        \item mit der Uni-ID anmelden
        \item Dokumente ausdrucken
    \end{enumerate}
\end{frame}

%%%%%%%%%%%%%%%%%%%%%%%%%%%%%%%%%%%%%%%%%%%%%%%%%%%%%%%%%%%%
% VPN
%%%%%%%%%%%%%%%%%%%%%%%%%%%%%%%%%%%%%%%%%%%%%%%%%%%%%%%%%%%%

\subsection{VPN}
\begin{frame}{Was ist denn VPN?}
    \large
    \begin{itemize}
        \item \textbf{V}irtual \textbf{P}rivate \textbf{N}etwork
        \item Ein VPN ermöglicht den (Fern-)Zugriff auf andere Netze
        \item Alle Daten werden für den Transport verschlüsselt
        \item hat erstmal nichts mit einem VPN Anbieter zu tun \\
              (NordVPN, TunnelBear, ExpressVPN)
    \end{itemize}
\end{frame}

\begin{frame}{Wozu nützt mir ein VPN?}
    \large
    \begin{itemize}
        \item sichere Verbindung (z.B. falls man im FlixBus o.ä. ist)
        \item man ist \glqq{}virtuell\grqq{} im Uni Netz (z.B. Passwort ändern, UB)
        \item man ist damit auch in Deutschland \\
            (z.B. Mediathek, (legales) Streaming)
        \item Nutzung von WLAN/Ethernet-Dosen
    \end{itemize}
    Download Link: \\
    {\normalsize \href{https://vpnsrv0.urz.uni-heidelberg.de/+CSCOE+/logon.html}{\texttt{vpnsrv0.urz.uni-heidelberg.de/+CSCOE+/logon.html}}}
\end{frame}

\begin{frame}{Wie sieht das aus?}
    \begin{center}
        \only<1>{%
            \vspace*{-1.1cm}
            \includegraphics[scale=2.3]{images/vpn_1.png}
        }
        \only<2>{%
            \vspace*{-0.7cm}
            \includegraphics[scale=2.3]{images/vpn.png}
        }
    \end{center}
\end{frame}

%%%%%%%%%%%%%%%%%%%%%%%%%%%%%%%%%%%%%%%%%%%%%%%%%%%%%%%%%%%%
% CIP-Pool
%%%%%%%%%%%%%%%%%%%%%%%%%%%%%%%%%%%%%%%%%%%%%%%%%%%%%%%%%%%%

\subsection{CIP-Pool}
\begin{frame}{CIP-Pool}
    \large
    \textbf{Es gibt verschiedene CIP-Pools:}
    \begin{itemize}
        \item Mathematikon Bibliothek (INF 205)
        \item Universitätsrechenzentrum (INF 293)
        \item Kichhoff Institut für Physik (INF 227)
    \end{itemize}
\end{frame}

%%%%%%%%%%%%%%%%%%%%%%%%%%%%%%%%%%%%%%%%%%%%%%%%%%%%%%%%%%%%
% Uni-Mail-Adresse
%%%%%%%%%%%%%%%%%%%%%%%%%%%%%%%%%%%%%%%%%%%%%%%%%%%%%%%%%%%%

\subsection{Uni-Mail-Adresse}
\begin{frame}{Uni-Mail-Adresse}
    \begin{itemize}
        \item<+-> jeder Studierende bekommt vom URZ eine Mail-Adresse \\
            \texttt{<name>@stud.uni-heidelberg.de} \\
            \texttt{<uni-id>@stud.uni-heidelberg.de} \\
        \item<+->diese Adresse ist abrufbar via Webinterface: \\
            \url{https://sogo.urz.uni-heidelberg.de} \\
        \item<+-> oder mit einem Mailclient (z.B. Thunderbird): \\
            \only<3>{
                \vspace{0.5em}
                \begin{tabular}{lll}
                    Aufgabe & Adresse & Port \\ \toprule
                    Posteingangsserver & \texttt{imap.urz.uni-heidelberg.de} & \texttt{993} \\
                    Postausgangsserver & \texttt{mail.urz.uni-heidelberg.de} & \texttt{587} \\
                    \href{https://www.urz.uni-heidelberg.de/de/uebersicht-e-mail-server}{\scriptsize Link zur Quelle} & & \\
                \end{tabular} \\
            }
        \item<+-> \emph{WICHTIG:} diese Mails solltet ihr lesen
        \item<+-> Man kann die Mails auf ein privates Konto weiterleiten \\
            SoGO: Preferences > Forward > Forward incoming messages (+ copy)
    \end{itemize}
\end{frame}

%%%%%%%%%%%%%%%%%%%%%%%%%%%%%%%%%%%%%%%%%%%%%%%%%%%%%%%%%%%%
% Wichtige Websites
%%%%%%%%%%%%%%%%%%%%%%%%%%%%%%%%%%%%%%%%%%%%%%%%%%%%%%%%%%%%

\section{Wichtige Websites}
\begin{frame}{Wichtige Websites}
    \large Die wichtigsten Websites im Unikontext sind:
    \begin{itemize}
        \item{LSF}
        \item{MÜSLI}
        \item{Moodle}
        \item{MaMpf}
    \end{itemize}
\end{frame}

%%%%%%%%%%%%%%%%%%%%%%%%%%%%%%%%%%%%%%%%%%%%%%%%%%%%%%%%%%%%
% LSF
%%%%%%%%%%%%%%%%%%%%%%%%%%%%%%%%%%%%%%%%%%%%%%%%%%%%%%%%%%%%

\subsection{LSF}
\begin{frame}{Das LSF -- Lehre, Studium und Forschung}

    \large \url{https://lsf.uni-heidelberg.de} \\
    \begin{minipage}[t]{0.515\textwidth}
        \begin{itemize}
            \item{Veranstaltungssuche}
            \item{Studienbescheinigung}
            \item{Rückmeldung}
            \item{Bafög-Daten}
            \item{Noten}
            \item{Stundenplan}
        \end{itemize}
    \end{minipage}
    \begin{minipage}[t]{0.4\textwidth}
        \vspace{0.4cm}
        \begin{center}
            \qrcode[height=0.65\textwidth, link]{https://lsf.uni-heidelberg.de}
        \end{center}
    \end{minipage}
\end{frame}

\begin{frame}{Veranstaltungssuche}
    \only<1>{
        Welche Veranstaltungen ihr belegen müsst/könnt, entnehmt ihr dem Modulhandbuch zu eurem Studiengang. Mehr dazu erfahrt ihr in den PO-Vorträgen am 7. Oktober um 14:00 Uhr.\\
        \vspace{0.5cm}
        \large {\LARGE \DejaSans{} ⚠} Achtung {\LARGE \DejaSans{}⚠} \\
        \normalsize
        \begin{itemize}
            \item{Nach (Pro-)Seminaren haltet ihr am besten in den Aushängen im Mathematikon oder im MÜSLI am Ende der Vorlesungszeit Ausschau (Vorbesprechungen finden teilweise in der vorletzten oder letzten Vorlesungswoche statt)}
            \item{Über Übungsgruppen informiert ihr euch im MÜSLI}
            \item{Für alles andere gibt es das LSF}
        \end{itemize}
    }
    \only<2>{
        \begin{figure}
            \centering
            \includegraphics[scale=0.3]{images/lsf01.jpg}
        \end{figure}
        Achtet darauf, dass das richtige Semester eingestellt ist, bevor ihr nach Veranstaltungen sucht.
    }
    \only<3>{
        \begin{figure}
            \centering
            \includegraphics[scale=0.25]{images/lsf02.jpg}
        \end{figure}
    }
    \only<4>{
        Zwei Möglichkeiten, im LSF nach Veranstaltungen zu suchen
        \begin{enumerate}
            \item Veranstaltungsangebot einer Fakultät durchsuchen \\
                \emph{Was wird angeboten?}
            \item Nach konkreter Veranstaltung über die Suchfunktion suchen \\
                \emph{Wird eine bestimmte Veranstaltung angeboten?} \\
                \emph{Wann und wo findet sie statt?}
        \end{enumerate}
    }
\end{frame}

\begin{frame}{Veranstaltungssuche über Fakultät -- Fach -- Studiengang}
    \only<1>
        \begin{figure}
            \centering
            \includegraphics[scale=0.25]{images/lsf04.jpg}
        \end{figure}
    }
    \only<3>
        \begin{figure}
            \centering
            \includegraphics[scale=0.25]{images/lsf06.jpg}
        \end{figure}
    }
    \only<5>
        \begin{figure}
            \centering
            \includegraphics[scale=0.25]{images/lsf08.jpg}
        \end{figure}
    }
\end{frame}

\begin{frame}{Veranstaltungssuche über die Suchfunktion}
    \only<1>{%
        \begin{figure}
            \centering
            \includegraphics[scale=0.25]{images/lsf09.jpg}
        \end{figure}
    }
    \only<2>{%
        \begin{figure}
           \centering
           \includegraphics[scale=0.3]{images/lsf10.jpg}
        \end{figure}
    }
\end{frame}

\begin{frame}{Studienbescheinigung, Bafög und Rückmeldung}
    Für diese Funktionen müsst ihr euch anmelden.
    \begin{figure}
        \centering
        \includegraphics[scale=0.3]{images/lsf12.jpg}
    \end{figure}
\end{frame}

\begin{frame}{Rückmeldung}
    \begin{itemize}
        \item{Erforderlich, wenn ihr weiter studieren wollt}
        \item{Rückmeldung = Semesterbeiträge bezahlen}
        \item{Aufforderung dazu: über die Uni-Mail-Adresse jeweils Mitte Januar und Mitte Juni}
        \item{Im \glqq Gebühren-Konto\grqq \ wird angezeigt, ob ihr rückgemeldet seid}
    \end{itemize}
\end{frame}

\begin{frame}{Noten}
    \begin{figure}
        \centering
        \includegraphics[scale=0.3]{images/lsf13.jpg}
    \end{figure}
    Eure Noten werden -- mit etwas Verzögerung -- vom jeweiligen fachinternen System hierher übertragen. Auch hierfür müsst ihr euch anmelden.
\end{frame}

\begin{frame}{Stundenplan}
    \begin{figure}
        \centering
        \includegraphics[scale=0.3]{images/lsf14.jpg}
    \end{figure}
    Achtung! Auf diese Weise meldet ihr euch NICHT für Kurse an.
\end{frame}

\begin{frame}{Studenplan}
    \begin{figure}
        \centering
        \includegraphics[scale=0.25]{images/lsf15.jpg}
    \end{figure}
\end{frame}

%%%%%%%%%%%%%%%%%%%%%%%%%%%%%%%%%%%%%%%%%%%%%%%%%%%%%%%%%%%%
% MÜSLI
%%%%%%%%%%%%%%%%%%%%%%%%%%%%%%%%%%%%%%%%%%%%%%%%%%%%%%%%%%%%

\subsection{MÜSLI}
\begin{frame}{MÜSLI -- \normalsize Mathematisches Übungsgruppen- und Scheinlisten-Interface}

    \large \url{https://muesli.mathi.uni-heidelberg.de/}

    \begin{minipage}[t]{0.7\textwidth}

    \begin{itemize}
        \item Eintragung in Übungsgruppen
        \item Einsehen von Zettelpunkten
        \item E-Mail-Adressen der Tutor*innen
        \item Klausuranmeldung
        \item Noten
    \end{itemize}
    \end{minipage}
    \begin{minipage}[t]{0.28\textwidth}
        \vspace*{0em}
        \begin{center}
            \qrcode[height=0.9\textwidth]{https://muesli.mathi.uni-heidelberg.de/}
        \end{center}
    \end{minipage}
\end{frame}

\begin{frame}{MÜSLI -- \normalsize Mathematisches Übungsgruppen- und Scheinlisten-Interface}
    \Large {\LARGE \DejaSans{} ⚠} Achtung {\LARGE \DejaSans{}⚠} \\
    \normalsize
    \begin{enumerate}
        \item{Ihr müsst euch mit eurer Uni-Mail-Adresse registrieren.}
        \item{In Mathe/Info: Anmeldung für Übungsgruppe = Anmeldung für Vorlesung}
        \item{In anderen Fächern gibt es andere Anmeldemodalitäten. Wenn ihr Kurse in einem anderen Fach belegen wollt/müsst, informiert euch frühzeitig(!) darüber, wie ihr euch in diesem Fach zu Veranstaltungen anmelden könnt.}
    \end{enumerate}
\end{frame}

%%%%%%%%%%%%%%%%%%%%%%%%%%%%%%%%%%%%%%%%%%%%%%%%%%%%%%%%%%%%
% Moodle
%%%%%%%%%%%%%%%%%%%%%%%%%%%%%%%%%%%%%%%%%%%%%%%%%%%%%%%%%%%%

\subsection{Moodle}
\begin{frame}{Moodle -- uniweite E-Learningplattform}

    Link: \url{https://elearning2.uni-heidelberg.de/}

    \begin{itemize}
        \item Anmeldung über Uni-ID
        \item{Kurse, in die ihr euch mit einem Einschreibeschlüssel eintragen könnt; Dozent*innen geben den Einschreibeschlüssel i.d.R. in der ersten Sitzung bekannt}
        \item{Materialien zu Veranstaltungen wie Übungsblätter oder Präsentationen}
        \item{Abgaben}
    \end{itemize}

\end{frame}

\begin{frame}{Moodle -- uniweite E-Learningplattform}
    \only<1>
        \begin{figure}
            \centering
            \includegraphics[scale=0.3]{images/moodle07.jpg}
        \end{figure}
    }
    \only<3>
        \begin{figure}
            \centering
            \includegraphics[scale=0.3]{images/moodle09.jpg}
        \end{figure}
    }
    \only<5>
    E-Mail-Adresse einstellen:\\
    \glqq Profil\grqq \ anklicken
        \begin{figure}
            \centering
            \includegraphics[scale=0.3]{images/moodle14.jpg}
        \end{figure}
    }
    \only<7>
    E-Mail-Adresse eingeben/ändern und Sichtbarkeiten anpassen
        \begin{figure}
            \centering
            \includegraphics[scale=0.3]{images/moodle16.jpg}
        \end{figure}
    }
    \only<9>
    \glqq Foren einstellen\grqq \ anklicken
        \begin{figure}
            \centering
            \includegraphics[scale=0.3]{images/moodle12.jpg}
        \end{figure}
    }
    \only<11>{%}
    Einstellungen anpassen
        \begin{figure}
            \centering
            \includegraphics[scale=0.3]{images/moodle13.jpg}
        \end{figure}
    }
\end{frame}

%%%%%%%%%%%%%%%%%%%%%%%%%%%%%%%%%%%%%%%%%%%%%%%%%%%%%%%%%%%%
% MaMpf
%%%%%%%%%%%%%%%%%%%%%%%%%%%%%%%%%%%%%%%%%%%%%%%%%%%%%%%%%%%%

\subsection{MaMpf}
\begin{frame}{\includegraphics[scale=0.072]{images/mampf.png} MaMpf -- Mathematische Medienplattform}

    Link: \url{https://mampf.mathi.uni-heidelberg.de/}

    \begin{itemize}
        \item Speziell fürs Mathelernen konzipierte E-Learning-Plattform
        \item Wird hier am Mathematischen Institut betrieben und entwickelt
        \item Viele Lernangebote (Vorlesungsvideos, Beispielsvideos, Quizzes, ...),\\
              die getaggt und untereinander vernetzt sind
        \item Werdet ihr benutzen, wenn ihr die LA1 hört
        \item Mehr zu MaMpf: \url{https://mampf.blog/}
    \end{itemize}
    \vspace{-3.7em}
    \begin{flushright}
        \qrcode[height=0.27\textwidth]{https://mampf.mathi.uni-heidelberg.de/}
    \end{flushright}


\end{frame}

\begin{frame}{\includegraphics[scale=0.072]{images/mampf.png} MaMpf -- Registrierung}
    \only<1>{
        Registriert euch einfach mit einer x-beliebigen E-Mail-Adresse.\\
        \begin{figure}
            \centering
            \includegraphics[scale=0.3]{images/mampf01.jpg}
        \end{figure}
    }
    \only<2>{
        MaMpf schickt euch an diese E-Mail-Adresse einen Aktivierungslink,
        den ihr anklicken müsst.\\
        \begin{figure}
            \centering
            \includegraphics[scale=0.3]{images/mampf02.jpg}
        \end{figure}
    }
\end{frame}

\begin{frame}{\includegraphics[scale=0.072]{images/mampf.png} MaMpf -- Profil einstellen}
    \only<1>{
        Und schon habt ihr ein MaMpf-Profil, das es einzurichten gilt.\\
        Wählt die Module aus, zu denen ihr Inhalte sehen wollt.
        \begin{figure}
            \centering
            \includegraphics[scale=0.2]{images/mampf03.jpg}
        \end{figure}
    }
    \only<2>{
        Manche Veranstaltungen gibt es mehrfach, weil sie regelmäßig stattfinden.\\
        Wählt eine primäre Veranstaltung (vorzugsweise die, die ihr hört),
        sonst enthält euch MaMpf Inhalte vor.\\
        \begin{figure}
            \centering
            \includegraphics[scale=0.3]{images/mampf25.jpg}
        \end{figure}
        Wenn ihr eine weitere Veranstaltung abonniert, könnt ihr auch auf deren
        Inhalte zugreifen.
        Dort findet ihr Erklärungen anderer Dozent*innen, die euch vielleicht
        weiterhelfen, wenn ihr etwas in eurer Vorlesung nicht versteht.
    }
    \only<3>{
        MaMpf informiert euch über Neuigkeiten (z.B. Mitteilungen von Dozent*innen
        und Obertutor*innen, Medien, Veranstaltungen, ...).\\
        Diese Informationen könnt ihr euch zusätzlich per E-Mail zuschicken lassen.
        \begin{figure}
            \centering
            \includegraphics[scale=0.3]{images/mampf04.jpg}
        \end{figure}
    }
\end{frame}

\begin{frame}{\includegraphics[scale=0.072]{images/mampf.png} MaMpf -- Tour}
      \only<1>{
              So sieht die Benachrichtigungsübersicht aus. Gerade ist nicht viel los.
              \begin{figure}
                    \centering
                    \includegraphics[scale=0.3]{images/mampf23.jpg}
              \end{figure}
      }
      \only<2>{
            Über das Drop-Down-Menü könnt ihr zwischen Veranstaltungen navigieren.
            \begin{figure}
                  \centering
                  \includegraphics[scale=0.25]{images/mampf07.jpg}
            \end{figure}
      }
      \only<3>{
            Auf der Veranstaltungsseite werden euch eine Inhaltsübersicht und
            Neuigkeiten zur Veranstaltung angezeigt, wie z.B. neue Forumsbeiträge.
            Von hier aus gelangt ihr zu Vorlesungsvideos und -mitschrieben,
            Übungsblätter, Beispielsvideos, Quizzes, ...
            \begin{figure}
                  \centering
                  \includegraphics[scale=0.3]{images/mampf16.jpg}
            \end{figure}
      }
      \only<4>{
            Wenn ihr eine weitere Veranstaltung im selben Modul abonniert habt,
            könnt ihr zwischen den Veranstaltungen hin- und herwechseln und gelangt
            so zu den Medien der jeweiligen Veranstaltung.
            \begin{figure}
                  \centering
                  \includegraphics[scale=0.3]{images/mampf26.jpg}
            \end{figure}
      }
      \only<5>{
            \textbf{Lektionen}: Hier könnt ihr euch Videos und Vorlesungsmitschriebe anschauen und herunterladen. Durch die Tags könnt ihr euch einen Überblick
            darüber verschaffen, worum es in den einzelnen Lektionen geht.
            \begin{figure}
                  \centering
                  \includegraphics[scale=0.3]{images/mampf08.jpg}
            \end{figure}
      }
      \only<6>{
            Wenn ihr Videos auf MaMpf schaut, werden sie im THymE-Player abgespielt.
            Dieser zeigt die Vorlesungsgliederung und Verweise an. Wenn ihr auf
            die Icons klickt, gelangt ihr zur referenzierten Stelle.
            \begin{figure}
                  \centering
                  \includegraphics[scale=0.3]{images/mampf19.jpg}
            \end{figure}
      }
      \only<7>{
            Hierher führt das Icon: Zum pdf-Dokument mit dem Beispiel, auf das im THymE-Player verwiesen wurde.
            \begin{figure}
                  \centering
                  \includegraphics[scale=0.17]{images/mampf20.jpg}
            \end{figure}
      }
      \only<8>{
            \textbf{Übungen}: Hier könnt ihr euch eure wöchtlich abzugebenden Übungszettel herunterladen.
            \begin{figure}
                  \centering
                  \includegraphics[scale=0.3]{images/mampf09.jpg}
            \end{figure}
      }
      \only<9>{
            \textbf{Quizzes}: Hier findet ihr zu bestimmten Themen zusammengestellte Quizzes, teilweise mit Erklärvideos.
            \begin{figure}
                  \centering
                  \includegraphics[scale=0.3]{images/mampf10.jpg}
            \end{figure}
      }
      \only<10>{
            \textbf{Selbsttest}: Hier könnt ihr Zufallsquizzes spielen, die ihr inhaltlich einschränken
            könnt, indem ihr etwas in das Inhaltsspezifikationsfeld eingebt.
            \begin{figure}
                  \centering
                  \includegraphics[scale=0.3]{images/mampf11.jpg}
            \end{figure}
      }
      \only<11>{
            Ihr könnt euch alle verfügbaren Tags anzeigen lassen, mit denen ihr
            die Quizzes einschränken könnt.
            \begin{figure}
                  \centering
                  \includegraphics[scale=0.3]{images/mampf12.jpg}
            \end{figure}
      }
      \only<12>{
            Wenn ihr etwas in das Inhaltsspezifikationsfeld eingebt, schlägt MaMpf
            euch Tags vor.
            \begin{figure}
                  \centering
                  \includegraphics[scale=0.3]{images/mampf13.jpg}
            \end{figure}
      }
      \only<13>{
            Ihr könnt das Quiz auch durch mehrere Tags einschränken.
            \begin{figure}
                  \centering
                  \includegraphics[scale=0.3]{images/mampf14.jpg}
            \end{figure}
      }
      \only<14>{
            Zufallsquiz, das ihr bekommt, wenn ihr auf \glqq Test beginnen\grqq \ klickt.
            \begin{figure}
                  \centering
                  \includegraphics[scale=0.4]{images/mampf15.jpg}
            \end{figure}
      }
      \only<15>{
            \textbf{Worked-Examples}: Hier findet ihr ausführliche Beispielvideos mit hilfreichen Tipps und Tricks.
            \begin{figure}
                  \centering
                  \includegraphics[scale=0.3]{images/mampf24.jpg}
            \end{figure}
      }
      \only<16>{
            \textbf{Wiederholung}: In weiterführenden Vorlesungen gibt es Wiederholungsvideos zu Grundlagen.
            \begin{figure}
                  \centering
                  \includegraphics[scale=0.3]{images/mampf17.jpg}
            \end{figure}
      }
      \only<17>{
            \textbf{Mitteilungen}: Hier findet ihr heraus, ob es etwas Neues zu eurer Veranstaltung gibt.
            \begin{figure}
                  \centering
                  \includegraphics[scale=0.3]{images/mampf18.jpg}
            \end{figure}
      }
      \only<18>{
            \textbf{Suche}: Ihr könnt MaMpf über das Suchfeld durchsuchen.
            \begin{figure}
                  \centering
                  \includegraphics[scale=0.3]{images/mampf21.jpg}
            \end{figure}
      }
      \only<19>{
            Tags haben ihre eigene Seite. Hier findet ihr heraus, mit welchen anderen
            Begriffen sie zusammenhängen und wo ihr mehr über sie erfahren könnt.
            \begin{figure}
                  \centering
                  \includegraphics[scale=0.25]{images/mampf22.jpg}
            \end{figure}
      }
\end{frame}

\begin{frame}{Übersicht: Die wichtigsten Websites}
    \large
    \begin{itemize}
        \item \textbf{LSF:} allgemeine Studiumsverwaltung
        \item \textbf{MÜSLI:} Studiumsverwaltung in Mathe und Info
        \item \textbf{Moodle:} uniweite E-Learning-Plattform
        \item \textbf{MaMpf:} E-Learning-Plattform für Mathe
    \end{itemize}
\end{frame}

%%%%%%%%%%%%%%%%%%%%%%%%%%%%%%%%%%%%%%%%%%%%%%%%%%%%%%%%%%%%
% HelpDesk Vorstellung
%%%%%%%%%%%%%%%%%%%%%%%%%%%%%%%%%%%%%%%%%%%%%%%%%%%%%%%%%%%%
\section{HelpDesk Vorstellung}
\begin{frame}{HelpDesk -- Hürden abbauen}
    \only<1-3>{%
        \large \textbf{Wofür sind wir da?}
        \normalsize
        \begin{itemize}
            \item<1-> Ich habe so grundlegende inhaltliche Fragen, dass ich sie im Tutorium gar nicht stellen kann oder will
                $\Rightarrow$ \textbf{fachlicher HelpDesk}
            \item<2-> Ich brauche Hilfe bei der Organisation meines Studiums, will aber nicht direkt mit einem Dozenten reden
                $\Rightarrow$ \textbf{Drumherum-HelpDesk}
        \end{itemize}
            \only<3->{%
                Seminarraum Statistik, 13-16 Uhr (außer Mittwoch) \\
                oder: per Mail an \mail{helpdesk@mathinf.uni-heidelberg.de}
            }
        }
    \only<4>{%
        \vspace{-1em}
        \includegraphics[width=1\textwidth]{help_desk.png}
    }
\end{frame}

%%%%%%%%%%%%%%%%%%%%%%%%%%%%%%%%%%%%%%%%%%%%%%%%%%%%%%%%%%%%
% Software/Anleitungen
%%%%%%%%%%%%%%%%%%%%%%%%%%%%%%%%%%%%%%%%%%%%%%%%%%%%%%%%%%%%

\section{Software \& Anleitungen}
\begin{frame}{Software/Anleitungen}
    Über das URZ kann man vergünstigt/kostenlos Software erwerben:
    \begin{itemize}
        \item Antivirensoftware \pause
        \item Microsoft Office 365 Pro Plus \pause
        \item Developement Software \pause
        \item Windows Lizenzen \pause
    \end{itemize}
    \url{https://www.urz.uni-heidelberg.de/de/lizenzmanagement}
\end{frame}

%%%%%%%%%%%%%%%%%%%%%%%%%%%%%%%%%%%%%%%%%%%%%%%%%%%%%%%%%%%%
% Universitätsbibliothek
%%%%%%%%%%%%%%%%%%%%%%%%%%%%%%%%%%%%%%%%%%%%%%%%%%%%%%%%%%%%

\section{Unibibliothek}
\begin{frame}{Unibibliothek}
    \only<1>{%
        Welche Bibliotheken gibt es an der Uni?
        \begin{itemize}
            \item \textbf{Institutsbibliotheken}
                \begin{itemize}
                    \item Fachspezifisch
                    \item Normalerweise nur Präsenzbibliotheken, d.h. keine Ausleihe möglich
                    \item Mathe-Info-Bib befindet sich im EG des Mathematikons
                \end{itemize}
            \item \textbf{Zentralbibliotheken}
                \begin{itemize}
                    \item  Nicht fachspezifisch
                    \item Hauptbibliothek in der Altstadt (Plöck 107-109, Ausleihe im EG)
                    \item Zweigstelle im Neuenheimer Feld (Im Neuenheimer Feld 368, Ausleihe im 3.OG)
                \end{itemize}
        \end{itemize}
    }
    \only<2>{%
        Beachtet
        \begin{itemize}
        \item \textbf{Bibliotheksangebot nutzen}
                \begin{itemize}
                    \item Studierendenausweis = Nutzerausweis
                    \item Um alle Angebote nutzen zu können, müsst ihr euren Ausweis freischalten \\
                        \href{https://www.ub.uni-heidelberg.de/service/anmeldung.html}{\texttt{www.ub.uni-heidelberg.de/service/anmeldung.html}}
                \end{itemize}
            \item \textbf{Schließfächer}
                \begin{itemize}
                    \item Zentralbibliotheken: Ihr braucht eine Zwei-Euro-Münze
                    \item Mathe-Info-Bib: Es gibt Schließfächer mit Zahlencodes oder Schlüsseln; Schließfachschüssel erhaltet ihr bei der Bibliotheksaufsicht gegen ein Pfand (z.B. Ausweis)
                \end{itemize}
        \end{itemize}
    }
    \only<3>{%
        Was bieten euch Bibliotheken?
        \begin{itemize}
            \item{\textbf{Lernort:} allein oder zusammen (Gruppenarbeitsräume reservieren: \url{https://www.ub.uni-heidelberg.de/service/gruppenarbeitsraeume.html})}
            \item{\textbf{Literatur}}
                  \begin{itemize}
                        \item{Finden, nutzen, ausleihen und downloaden}
                        \item{Mehr Literatur über Datenbanken finden}
                        \item{Fernleihe}
                        \item{Teilweise Zugriff auf Online-Angebot von Zeitschriften}
                  \end{itemize}
            \item Weitere Informationen (z.B. Führungen und Kurse) \\
                \href{https://www.ub.uni-heidelberg.de/schulung/Welcome.html}{\texttt{www.ub.uni-heidelberg.de/schulung/Welcome.html}}
        \end{itemize}
    }
\end{frame}

\begin{frame}{Unibiliothek -- HEIDI}
    \only<1>{%}
        Bibliothekskatolog HEIDI
        \begin{itemize}
            \item \url{https://katalog.ub.uni-heidelberg.de}
            \item Medien finden und ihren Standort in Erfahrung bringen oder downloaden
        \end{itemize}
    }
    \only<2>{
        \begin{figure}
            \centering
            \includegraphics[scale=0.35]{images/ub01.jpg}
        \end{figure}
    }
    \only<3>{
        \begin{figure}
            \centering
            \includegraphics[scale=0.2]{images/ub02.jpg}
        \end{figure}
    }
    \only<4>{
        \begin{figure}
            \centering
            \includegraphics[scale=0.2]{images/ub03.jpg}
        \end{figure}
    }
    \only<5>{
        \begin{figure}
            \centering
            \includegraphics[scale=0.2]{images/ub04.jpg}
        \end{figure}
    }
\end{frame}

%%%%%%%%%%%%%%%%%%%%%%%%%%%%%%%%%%%%%%%%%%%%%%%%%%%%%%%%%%%%
% Fachschaftsservices
%%%%%%%%%%%%%%%%%%%%%%%%%%%%%%%%%%%%%%%%%%%%%%%%%%%%%%%%%%%%

\section{Fachschaftsservices}
\begin{frame}{Fachschaftsservices}
    \begin{itemize}[<+->]
        \item \textbf{Unsere Webseite:} \href{mathphys.info}{\texttt{mathphys.info}} \\
            Veranstaltungshinweise, Informationen über Gremienmitglieder
        \item \textbf{Der Kummerkasten:} \href{https://kummerkasten.mathphys.info}{\texttt{kummerkasten.mathphys.info}} \\
            anonyme Rückmeldung an Dozenten der Vorlesungen
        \item \textbf{Klausurenordner:} \\
            Im Fachschaftsraum (INF 205 // 01.301) gibt es alte Klasuren
        \item \textbf{Mailinglisten:}
            \begin{itemize}
                \item[--] \mail{fachschaft@mathphys.stura.uni-heidelberg.de} \\
                \item[--] \mail{sos@mathphys.stura.uni-heidelberg.de} \\
            \end{itemize}
    \end{itemize}
\end{frame}

\begin{frame}{Vielen Dank fürs Zuhören!}
    \vfill
    \begin{center}
        \Huge Fragen \\[3.5pt]
        \normalsize Serviceangebote Studium
    \end{center}
\end{frame}

\end{document}
