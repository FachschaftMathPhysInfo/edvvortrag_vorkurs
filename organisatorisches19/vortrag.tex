%pdf-latex

%header nur vom MÜSLI-Vortrag geklaut
% !TEX root = present.tex
% !TEX encoding = UTF-8 Unicode

\documentclass[
    t,
    ngerman,
    hyperref={
        colorlinks=false
    }
    ]{beamer} %compress,

%% Pakete laden...
  \usepackage[T1]{fontenc}
  \usepackage[utf8]{inputenc}
  \usepackage{
      babel,
      bookmark,
      booktabs,
%      blindtext,
      colortbl,
%      eurosym,
      graphicx,
      hyperref,
%      libertine,
      microtype,
      pifont,
      pgfpages,
      tikz,
      wrapfig,
      qrcode,
%      xspace,
  }


%% Design festlegen...
  \mode<presentation>{
%      \useoutertheme[subsection=false]{smoothbars}
      \useinnertheme{rectangles} % rectangles, circles, rounded
      \usecolortheme[RGB={153,0,0}]{structure}
      \definecolor{unihd}{RGB}{153,0,0}
      \definecolor{dark}{RGB}{115,0,0}
      \definecolor{light}{RGB}{241,229,229}
      \usecolortheme{whale}
         \usecolortheme{orchid}
%      \usecolortheme{beaver}
%      \setbeamercovered{transparent}
      \beamertemplatenavigationsymbolsempty
%      \setbeameroption{show notes on second screen}
      \setbeamertemplate{note page}[plain]
      \logo{\includegraphics[width=3.5cm]{./media/fs-logo-small.pdf}}

  }

%% nützliche Definitionen...
  \graphicspath{{media/}}

%% Titelinformationen...
  \title[EDV]{Serviceangebote Studium\\(VPN, eduroam, URZ, drucken, UB)}
  \author[
    Patrick und Chris
  ]{
    Patrick Dammann\\{\scriptsize\url{patrick@mathphys.stura.uni-heidelberg.de}}\\
    Christian Heusel\\{\scriptsize\url{chris@mathphys.stura.uni-heidelberg.de}}
}
%  \institute{  \includegraphics[width=5.5cm]{fs-logo-big} }

  \date[02.10.2017]{\vspace*{-2em}\\ 02. Oktober 2018}

  \hypersetup{
      pdfauthor={Patrick Damann, Christian Heusel},
      pdftitle={Serviceangebote Studium},
      pdfsubject={Serviceangebote Studium},
      pdfkeywords={1},
      pdfpagelayout={SinglePage},
  }



\newenvironment{rcases}{%
  \left.\renewcommand*\lbrace.%
  \begin{cases}}%
{\end{cases}\right\rbrace}


\begin{document}

\begin{frame}[fragile]
    \maketitle{}
\end{frame}

\begin{frame}{Inhalt des Vortrags}
    \begin{minipage}[t]{0.515\textwidth}
        \tableofcontents[hideallsubsections, sections={1-5}]
    \end{minipage}
    \begin{minipage}[t]{0.475\textwidth}
        \tableofcontents[hideallsubsections, sections={6-11}]
    \end{minipage}
\end{frame}

\begin{frame}{Wer will mir hier was erzählen!?}
    \vfill
    \begin{center}
        {\Large Janina Rastetter} \\
        \mail{jrastetter@mathi.uni-heidelberg.de} \\
        \vspace{1em}
        {\Large Christian Heusel } \\
        \mail{chris@mathphys.stura.uni-heidelberg.de} \\
    \end{center}
\end{frame}

\begin{frame}{Präambel}
    \Large
    \begin{center}
        \only<1->{%
            {\huge \DejaSans{} 😇} Es wird die Folien zum Download geben! {\huge \DejaSans{} 😇}\\[0.7em]
        }
        \only<1>{\includegraphics[scale=0.6]{images/xkcd_1.png}}
        \only<2->{%
            {\huge \DejaSans{} 😺} Stellt Fragen, im Währenden oder am Ende! {\huge \DejaSans{} 😺}\\[0.7em]
        }
        \only<3->{%
            {\huge \DejaSans{} 😱} Es wird viele Informationen geben! {\huge \DejaSans{} 😱}\\[0.7em]
        }
        \only<4->{%
            \includegraphics[]{images/lets_go.png}
        }
    \end{center}
\end{frame}

%%%%%%%%%%%%%%%%%%%%%%%%%%%%%%%%%%%%%%%%%%%%%%%%%%%%%%%%%%%%
% Digitale Infrastruktur
%%%%%%%%%%%%%%%%%%%%%%%%%%%%%%%%%%%%%%%%%%%%%%%%%%%%%%%%%%%%

\section{Digitale Infrastruktur}
\begin{frame}{Digitale Infrastruktur}
    \large
    \begin{itemize}
        \item{WLAN}
        \item{Drucken}
        \item{VPN}
        \item{CIP-Pool}
        \item{Uni-Mail-Adresse}
    \end{itemize}
\end{frame}

%%%%%%%%%%%%%%%%%%%%%%%%%%%%%%%%%%%%%%%%%%%%%%%%%%%%%%%%%%%%
% WLAN
%%%%%%%%%%%%%%%%%%%%%%%%%%%%%%%%%%%%%%%%%%%%%%%%%%%%%%%%%%%%

\subsection{WLAN}
\begin{frame}{Eduroam}
\end{frame}

\begin{frame}{Eduroam einrichten -- leicht gemacht}
\end{frame}

%%%%%%%%%%%%%%%%%%%%%%%%%%%%%%%%%%%%%%%%%%%%%%%%%%%%%%%%%%%%
% Drucken
%%%%%%%%%%%%%%%%%%%%%%%%%%%%%%%%%%%%%%%%%%%%%%%%%%%%%%%%%%%%

\subsection{Drucken}
\begin{frame}{Drucken}
\end{frame}

%%%%%%%%%%%%%%%%%%%%%%%%%%%%%%%%%%%%%%%%%%%%%%%%%%%%%%%%%%%%
% VPN
%%%%%%%%%%%%%%%%%%%%%%%%%%%%%%%%%%%%%%%%%%%%%%%%%%%%%%%%%%%%

\subsection{VPN}
\begin{frame}{Was ist denn VPN?}
    \large
    \begin{itemize}
        \item \textbf{V}irtual \textbf{P}rivate \textbf{N}etwork
        \item Ein VPN ermöglicht den (Fern-)Zugriff auf andere Netze
        \item Alle Daten werden für den Transport verschlüsselt
        \item hat erstmal nichts mit einem VPN Anbieter zu tun \\
              (NordVPN, TunnelBear, ExpressVPN)
        \item bla bla bla
    \end{itemize}
\end{frame}

\begin{frame}{Wozu nützt mir ein VPN?}
\end{frame}

\begin{frame}{Wie sieht das aus?}
    \begin{center}
        \only<1>{%
            \vspace*{-1.1cm}
            \includegraphics[scale=2.3]{images/vpn_1.png}
        }
        \only<2>{%
            \vspace*{-0.7cm}
            \includegraphics[scale=2.3]{images/vpn.png}
        }
    \end{center}
\end{frame}

%%%%%%%%%%%%%%%%%%%%%%%%%%%%%%%%%%%%%%%%%%%%%%%%%%%%%%%%%%%%
% CIP-Pool
%%%%%%%%%%%%%%%%%%%%%%%%%%%%%%%%%%%%%%%%%%%%%%%%%%%%%%%%%%%%

\subsection{CIP-Pool}
\begin{frame}{CIP-Pool}
\end{frame}

%%%%%%%%%%%%%%%%%%%%%%%%%%%%%%%%%%%%%%%%%%%%%%%%%%%%%%%%%%%%
% Uni-Mail-Adresse
%%%%%%%%%%%%%%%%%%%%%%%%%%%%%%%%%%%%%%%%%%%%%%%%%%%%%%%%%%%%

\subsection{Uni-Mail-Adresse}
\begin{frame}{Uni-Mail-Adresse}
\end{frame}

%%%%%%%%%%%%%%%%%%%%%%%%%%%%%%%%%%%%%%%%%%%%%%%%%%%%%%%%%%%%
% Wichtige Websites
%%%%%%%%%%%%%%%%%%%%%%%%%%%%%%%%%%%%%%%%%%%%%%%%%%%%%%%%%%%%

\section{Wichtige Websites}
\begin{frame}{Wichtige Websites}
    \large Die wichtigsten Websites im Unikontext sind:
    \begin{itemize}
        \item{LSF}
        \item{MÜSLI}
        \item{Moodle}
        \item{MaMpf}
    \end{itemize}
\end{frame}

%%%%%%%%%%%%%%%%%%%%%%%%%%%%%%%%%%%%%%%%%%%%%%%%%%%%%%%%%%%%
% LSF
%%%%%%%%%%%%%%%%%%%%%%%%%%%%%%%%%%%%%%%%%%%%%%%%%%%%%%%%%%%%

\subsection{LSF}
\begin{frame}{Das LSF -- Lehre, Studium und Forschung}

    \large \url{https://lsf.uni-heidelberg.de} \\
    \begin{minipage}[t]{0.515\textwidth}
        \begin{itemize}
            \item{Veranstaltungssuche}
            \item{Studienbescheinigung}
            \item{Rückmeldung}
            \item{Bafög-Daten}
            \item{Noten}
            \item{Stundenplan}
        \end{itemize}
    \end{minipage}
    \begin{minipage}[t]{0.4\textwidth}
        \vspace{0.4cm}
        \begin{center}
            \qrcode[height=0.65\textwidth, link]{https://lsf.uni-heidelberg.de}
        \end{center}
    \end{minipage}
\end{frame}

\begin{frame}{Veranstaltungssuche}
    \only<1>{
          Welche Veranstaltungen ihr belegen müsst/könnt, entnehmt ihr dem Modulhandbuch zu eurem Studiengang. Dazu wird es einen gesonderten Vortrag geben.\\
          \vspace{0.5cm}
          Achtung!\\
          \begin{itemize}
                \item{Nach (Pro-)Seminaren haltet ihr am besten in den Aushängen im Mathematikon oder im MÜSLI am Ende der Vorlesungszeit Ausschau (Vorbesprechungen finden teilweise in der vorletzten oder letzten Vorlesungswoche statt)}
                \item{Über Übungsgruppen informiert ihr euch im MÜSLI}
                \item{Für alles andere gibt es das LSF}
          \end{itemize}
    }
    \only<2>{
          \begin{figure}
                \centering
                \includegraphics[scale=0.3]{images/lsf01.jpg}
          \end{figure}
          Achtet darauf, dass das richtige Semester eingestellt ist, bevor ihr nach Veranstaltungen sucht.
    }
    \only<3>{
          \begin{figure}
                \centering
                \includegraphics[scale=0.25]{images/lsf02.jpg}
          \end{figure}
    }
    \only<4>{
          Zwei Möglichkeiten, im LSF nach Veranstaltungen zu suchen
          \begin{enumerate}
                \item{Veranstaltungsangebot einer Fakultät durchsuchen (Was wird angeboten?)}
                \item{Nach konkreter Veranstaltung über die Suchfunktion suchen (Wird eine bestimmte Veranstaltung angeboten? Wann und wo findet sie statt?)}
          \end{enumerate}
    }
\end{frame}

\begin{frame}{Veranstaltungssuche über Fakultät -- Fach -- Studiengang}
    \only<1>
        \begin{figure}
            \centering
            \includegraphics[scale=0.25]{images/lsf04.jpg}
        \end{figure}
    }
    \only<3>
        \begin{figure}
            \centering
            \includegraphics[scale=0.25]{images/lsf06.jpg}
        \end{figure}
    }
    \only<5>
        \begin{figure}
            \centering
            \includegraphics[scale=0.25]{images/lsf08.jpg}
        \end{figure}
    }
\end{frame}

\begin{frame}{Veranstaltungssuche über die Suchfunktion}
    \only<1>{%
        \begin{figure}
            \centering
            \includegraphics[scale=0.25]{images/lsf09.jpg}
        \end{figure}
    }
    \only<2>{%
        \begin{figure}
           \centering
           \includegraphics[scale=0.3]{images/lsf10.jpg}
        \end{figure}
    }
\end{frame}

\begin{frame}{Studienbescheinigung, Bafög und Rückmeldung}
    Für diese Funktionen müsst ihr euch anmelden.
    \begin{figure}
        \centering
        \includegraphics[scale=0.3]{images/lsf12.jpg}
    \end{figure}
\end{frame}

\begin{frame}{Rückmeldung}
    \begin{itemize}
        \item{Erforderlich, wenn ihr weiter studieren wollt}
        \item{Rückmeldung = Semesterbeiträge bezahlen}
        \item{Aufforderung dazu: über die Uni-Mail-Adresse jeweils Mitte Januar und Mitte Juni}
    \end{itemize}
\end{frame}

\begin{frame}{Noten}
    \begin{figure}
        \centering
        \includegraphics[scale=0.3]{images/lsf13.jpg}
    \end{figure}
    Eure Noten werden -- mit etwas Verzögerung -- vom jeweiligen fachinternen System hierher übertragen. Auch hierfür müsst ihr euch anmelden.
\end{frame}

\begin{frame}{Stundenplan}
    \begin{figure}
        \centering
        \includegraphics[scale=0.3]{images/lsf14.jpg}
    \end{figure}
    Achtung! Auf diese Weise meldet ihr euch NICHT für Kurse an.
\end{frame}

\begin{frame}{Studenplan}
    \begin{figure}
        \centering
        \includegraphics[scale=0.25]{images/lsf15.jpg}
    \end{figure}
\end{frame}

%%%%%%%%%%%%%%%%%%%%%%%%%%%%%%%%%%%%%%%%%%%%%%%%%%%%%%%%%%%%
% MÜSLI
%%%%%%%%%%%%%%%%%%%%%%%%%%%%%%%%%%%%%%%%%%%%%%%%%%%%%%%%%%%%

\subsection{MÜSLI}
\begin{frame}{MÜSLI -- \normalsize Mathematisches Übungsgruppen- und Scheinlisten-Interface}

    \large \url{https://muesli.mathi.uni-heidelberg.de/}

    \begin{minipage}[t]{0.7\textwidth}

    \begin{itemize}
        \item Eintragung in Übungsgruppen
        \item Einsehen von Zettelpunkten
        \item E-Mail-Adressen der Tutor*innen
        \item Klausuranmeldung
        \item Noten
    \end{itemize}
    \end{minipage}
    \begin{minipage}[t]{0.28\textwidth}
        \vspace*{0em}
        \begin{center}
            \qrcode[height=0.9\textwidth]{https://muesli.mathi.uni-heidelberg.de/}
        \end{center}
    \end{minipage}
\end{frame}

\begin{frame}{MÜSLI -- \normalsize Mathematisches Übungsgruppen- und Scheinlisten-Interface}
    \Large {\LARGE \DejaSans{} ⚠} Achtung {\LARGE \DejaSans{}⚠} \\
    \normalsize
    \begin{enumerate}
        \item{Ihr müsst euch mit eurer Uni-Mail-Adresse registrieren.}
        \item{In anderen Fächern gibt es andere Anmeldemodalitäten. Wenn ihr Kurse in einem anderen Fach belegen wollt/müsst, informiert euch frühzeitig(!) darüber, wie ihr euch in diesem Fach zu Veranstaltungen anmelden könnt.}
    \end{enumerate}
\end{frame}

%%%%%%%%%%%%%%%%%%%%%%%%%%%%%%%%%%%%%%%%%%%%%%%%%%%%%%%%%%%%
% Moodle
%%%%%%%%%%%%%%%%%%%%%%%%%%%%%%%%%%%%%%%%%%%%%%%%%%%%%%%%%%%%

\subsection{Moodle}
\begin{frame}{Moodle -- uniweite E-Learningplattform}

    Link: \url{https://elearning2.uni-heidelberg.de/}

    \begin{center}
        \qrcode{https://elearning2.uni-heidelberg.de/}
    \end{center}

    \begin{itemize}
        \item Anmeldung über Uni-ID
        \item{Kurse, in die ihr euch mit einem Einschreibeschlüssel eintragen könnt; Dozent*innen geben den Einschreibeschlüssel i.d.R. in der ersten Sitzung bekannt}
        \item{Materialien zu Veranstaltungen wie Übungsblätter oder Präsentationen}
        \item{Abgaben}
    \end{itemize}

\end{frame}

\begin{frame}{Moodle -- uniweite E-Learningplattform}
    \only<1>
        \begin{figure}
            \centering
            \includegraphics[scale=0.3]{images/moodle02.jpg}
        \end{figure}
    }
    \only<3>
        \begin{figure}
            \centering
            \includegraphics[scale=0.3]{images/moodle04.jpg}
        \end{figure}
    }
    \only<5>{%}
        \begin{figure}
            \centering
            \includegraphics[scale=0.3]{images/moodle05.jpg}
        \end{figure}
    }
\end{frame}

%%%%%%%%%%%%%%%%%%%%%%%%%%%%%%%%%%%%%%%%%%%%%%%%%%%%%%%%%%%%
% MaMpf
%%%%%%%%%%%%%%%%%%%%%%%%%%%%%%%%%%%%%%%%%%%%%%%%%%%%%%%%%%%%

\subsection{MaMpf}
\begin{frame}{\includegraphics[scale=0.072]{images/mampf.png} MaMpf -- Mathematische Medienplattform}

    Link: \url{https://mampf.mathi.uni-heidelberg.de/}

    \begin{center}
        \qrcode{https://mampf.mathi.uni-heidelberg.de/}
    \end{center}

    \begin{itemize}
        \item Speziell fürs Mathelernen entwickelte E-Learning-Plattform
        \item Viele Lernangebote (Vorlesungsvideos, Beispielsvideos, Quizzes, ...)
        \item Lernangebote sind getaggt und untereinander vernetzt
        \item Werdet ihr benutzen, wenn ihr die LA1 hört
    \end{itemize}

\end{frame}

\begin{frame}{\includegraphics[scale=0.072]{images/mampf.png} MaMpf -- Registrierung}
      \only<1>{
            Registriert euch einfach mit einer x-beliebigen E-Mail-Adresse.\\
            \begin{figure}
                  \centering
                  \includegraphics[scale=0.3]{images/mampf01.jpg}
            \end{figure}
      }
      \only<2>{
            MaMpf schickt euch an diese E-Mail-Adresse einen Aktivierungslink,
            den ihr anklicken müsst.\\
            \begin{figure}
                  \centering
                  \includegraphics[scale=0.3]{images/mampf02.jpg}
            \end{figure}
      }
\end{frame}

\begin{frame}{\includegraphics[scale=0.072]{images/mampf.png} MaMpf -- Profil einstellen}
      \only<1>{
            Und schon habt ihr ein MaMpf-Profil, das es einzurichten gilt.\\
            Wählt die Module aus, zu denen ihr Inhalte sehen wollt.
            \begin{figure}
                  \centering
                  \includegraphics[scale=0.2]{images/mampf03.jpg}
            \end{figure}
      }
      \only<2>{
            Manche Veranstaltungen gibt es mehrfach, weil sie regelmäßig stattfinden.\\
            Passt also auf, dass ihr die richtige in eurem Profil ausgewählt habt.\\
            Eure LA1 wird in den nächsten Tagen freigeschaltet.
            \begin{figure}
                  \centering
                  \includegraphics[scale=0.4]{images/mampf06.jpg}
            \end{figure}
      }
      \only<3>{
            MaMpf informiert euch über Neuigkeiten (z.B. Mitteilungen von Dozent*innen
            und Obertutor*innen, Medien, Veranstaltungen, ...).\\
            Diese Informationen könnt ihr euch zusätzlich per E-Mail zuschicken lassen.
            \begin{figure}
                  \centering
                  \includegraphics[scale=0.3]{images/mampf04.jpg}
            \end{figure}
      }
      \only<4>{
            So sieht die Benachrichtigungsübersicht aus. Gerade ist nicht viel los.
            \begin{figure}
                  \centering
                  \includegraphics[scale=0.3]{images/mampf23.jpg}
            \end{figure}
      }
\end{frame}

\begin{frame}{\includegraphics[scale=0.072]{images/mampf.png} MaMpf -- Tour}
      \only<1>{
            Über das Drop-Down-Menü könnt ihr zwischen Veranstaltungen navigieren.
            \begin{figure}
                  \centering
                  \includegraphics[scale=0.25]{images/mampf07.jpg}
            \end{figure}
      }
      \only<2>{
            Auf der Veranstaltungsseite werden euch eine Inhaltsübersicht und
            Neuigkeiten zur Veranstaltung angezeigt, wie z.B. neue Forumsbeiträge.
            Von hier aus gelangt ihr zu Vorlesungsvideos und -mitschrieben,
            Übungsblätter, Beispielsvideos, Quizzes, ...
            \begin{figure}
                  \centering
                  \includegraphics[scale=0.3]{images/mampf16.jpg}
            \end{figure}
      }
      \only<3>{
            \textbf{Lektionen}: Hier könnt ihr euch Videos und Vorlesungsmitschriebe anschauen und herunterladen. Durch die Tags könnt ihr euch einen Überblick
            darüber verschaffen, worum es in den einzelnen Lektionen geht.
            \begin{figure}
                  \centering
                  \includegraphics[scale=0.3]{images/mampf08.jpg}
            \end{figure}
      }
      \only<4>{
            Wenn ihr Videos auf MaMpf schaut, werden sie im THymE-Player abgespielt.
            Dieser zeigt die Vorlesungsgliederung und Verweise an. Wenn ihr auf
            die Icons klickt, gelangt ihr zur referenzierten Stelle.
            \begin{figure}
                  \centering
                  \includegraphics[scale=0.3]{images/mampf19.jpg}
            \end{figure}
      }
      \only<5>{
            Hierher führt das Icon: Zum pdf-Dokument mit dem Beispiel, auf das im THymE-Player verwiesen wurde.
            \begin{figure}
                  \centering
                  \includegraphics[scale=0.17]{images/mampf20.jpg}
            \end{figure}
      }
      \only<6>{
            \textbf{Übungen}: Getaggte Übungsblätter. Hier könnt ihr euch eure wöchtlich abzugebenden Übungszettel herunterladen.
            \begin{figure}
                  \centering
                  \includegraphics[scale=0.3]{images/mampf09.jpg}
            \end{figure}
      }
      \only<7>{
            \textbf{Quizzes}: Hier findet ihr zu bestimmten Themen zusammengestellte Quizzes, teilweise mit Erklärvideos.
            \begin{figure}
                  \centering
                  \includegraphics[scale=0.3]{images/mampf10.jpg}
            \end{figure}
      }
      \only<8>{
            \textbf{Selbsttest}: Hier könnt ihr Zufallsquizzes spielen, die ihr inhaltlich einschränken
            könnt, indem ihr etwas in das Inhaltsspezifikationsfeld eingebt.
            \begin{figure}
                  \centering
                  \includegraphics[scale=0.3]{images/mampf11.jpg}
            \end{figure}
      }
      \only<9>{
            Ihr könnt euch alle verfügbaren Tags anzeigen lassen, mit denen ihr
            die Quizzes einschränken könnt.
            \begin{figure}
                  \centering
                  \includegraphics[scale=0.3]{images/mampf12.jpg}
            \end{figure}
      }
      \only<10>{
            Wenn ihr etwas in das Inhaltsspezifikationsfeld eingebt, schlägt MaMpf
            euch Tags vor.
            \begin{figure}
                  \centering
                  \includegraphics[scale=0.3]{images/mampf13.jpg}
            \end{figure}
      }
      \only<11>{
            Ihr könnt das Quiz auch durch mehrere Tags einschränken.
            \begin{figure}
                  \centering
                  \includegraphics[scale=0.3]{images/mampf14.jpg}
            \end{figure}
      }
      \only<12>{
            Zufallsquiz, das ihr bekommt, wenn ihr auf \glqq Test beginnen\grqq \ klickt.
            \begin{figure}
                  \centering
                  \includegraphics[scale=0.3]{images/mampf15.jpg}
            \end{figure}
      }
      \only<13>{
            \textbf{Worked-Examples}: Hier findet ihr ausführliche Beispielvideos mit hilfreichen Tipps und Tricks.
            \begin{figure}
                  \centering
                  \includegraphics[scale=0.3]{images/mampf24.jpg}
            \end{figure}
      }
      \only<14>{
            \textbf{Wiederholung}: In weiterführenden Vorlesungen gibt es Wiederholungsvideos zu Grundlagen.
            \begin{figure}
                  \centering
                  \includegraphics[scale=0.3]{images/mampf17.jpg}
            \end{figure}
      }
      \only<15>{
            \textbf{Mitteilungen}: Hier findet ihr heraus, ob es etwas Neues zu eurer Veranstaltung gibt.
            \begin{figure}
                  \centering
                  \includegraphics[scale=0.3]{images/mampf18.jpg}
            \end{figure}
      }
\end{frame}

\begin{frame}{\includegraphics[scale=0.072]{images/mampf.png} MaMpf -- Suche und Nagivation}
      \only<1>{
            Ihr könnt MaMpf über das Suchfeld durchsuchen.
            \begin{figure}
                  \centering
                  \includegraphics[scale=0.3]{images/mampf21.jpg}
            \end{figure}
      }
      \only<2>{
            Tags haben ihre eigene Seite. Hier findet ihr heraus, mit welchen anderen
            Begriffen sie zusammenhängen und wo ihr mehr über sie erfahren könnt.
            \begin{figure}
                  \centering
                  \includegraphics[scale=0.25]{images/mampf22.jpg}
            \end{figure}
      }
\end{frame}

\begin{frame}{Die wichtigsten Websites -- Zusammenfassung}
    \begin{itemize}
        \item LSF: allgemeine Studiumsverwaltung
        \item MÜSLI: Studiumsverwaltung in Mathe und Info
        \item Moodle: uniweite E-Learning-Plattform
        \item MaMpf: E-Learning-Plattform für Mathe
    \end{itemize}
\end{frame}

%%%%%%%%%%%%%%%%%%%%%%%%%%%%%%%%%%%%%%%%%%%%%%%%%%%%%%%%%%%%
% Software/Anleitungen
%%%%%%%%%%%%%%%%%%%%%%%%%%%%%%%%%%%%%%%%%%%%%%%%%%%%%%%%%%%%

\section{Software \& Anleitungen}
\begin{frame}{Software/Anleitungen}
\end{frame}

%%%%%%%%%%%%%%%%%%%%%%%%%%%%%%%%%%%%%%%%%%%%%%%%%%%%%%%%%%%%
% Universitätsbibliothek
%%%%%%%%%%%%%%%%%%%%%%%%%%%%%%%%%%%%%%%%%%%%%%%%%%%%%%%%%%%%

\section{Unibibliothek}
\begin{frame}{Unibibliothek}
    \only<1>{%
        Welche Bibliotheken gibt es an der Uni?
        \begin{itemize}
            \item Institutsbibliotheken
                \begin{itemize}
                    \item Fachspezifisch
                    \item Normalerweise nur Präsenzbibliotheken, d.h. keine Ausleihe möglich
                    \item Mathe-Info-Bib befindet sich im EG des Mathematikons
                \end{itemize}
            \item Zentralbibliotheken
                \begin{itemize}
                    \item  Nicht fachspezifisch
                    \item Hauptbibliothek in der Altstadt (Plöck 107-109, Ausleihe im EG)
                    \item Zweigstelle im Neuenheimer Feld (Im Neuenheimer Feld 368, Ausleihe im 3.OG)
                \end{itemize}
        \end{itemize}
    }
    \only<2>{%
        Beachtet
        \begin{itemize}
            \item {Bibliotheksangebot nutzen}
                \begin{itemize}
                    \item Studierendenausweis = Nutzerausweis
                    \item{Um alle Angebote nutzen zu können, müsst ihr euren Ausweis freischalten \url{https://www.ub.uni-heidelberg.de/service/anmeldung.html}}
                \end{itemize}
            \item Schließfächer
                \begin{itemize}
                    \item Zentralbibliotheken: Ihr braucht eine Zwei-Euro-Münze
                    \item Mathe-Info-Bib: Es gibt Schließfächer mit Zahlencodes oder Schlüsseln; Schließfachschüssel erhaltet ihr bei der Bibliotheksaufsicht gegen ein Pfand (z.B. Ausweis)
                \end{itemize}
        \end{itemize}
    }
    \only<3>{%
        Was bieten euch Bibliotheken?
        \begin{itemize}
            \item{Lernort: allein oder zusammen (Gruppenarbeitsräume reservieren: \url{https://www.ub.uni-heidelberg.de/service/gruppenarbeitsraeume.html})}
            \item{Literatur}
                  \begin{itemize}
                        \item{Finden, nutzen, ausleihen und downloaden}
                        \item{Mehr Literatur über Datenbanken finden}
                        \item{Fernleihe}
                        \item{Teilweise Zugriff auf Online-Angebot von Zeitschriften}
                  \end{itemize}
            \item{Weitere Informationen (z.B. Führungen und Kurse) \url{https://www.ub.uni-heidelberg.de/schulung/Welcome.html}}
        \end{itemize}
    }
\end{frame}

\begin{frame}{Unibiliothek -- HEIDI}
      \only<1>{%}
          Bibliothekskatolog HEIDI
          \begin{itemize}
                \item \url{https://katalog.ub.uni-heidelberg.de/cgi-bin/search.cgi?zweig=}
                \item Medien finden und ihren Standort in Erfahrung bringen oder downloaden
          \end{itemize}
      }
      \only<2>{
            \begin{figure}
                \centering
                \includegraphics[scale=0.35]{images/ub01.jpg}
            \end{figure}
      }
      \only<3>{
            \begin{figure}
                \centering
                \includegraphics[scale=0.2]{images/ub02.jpg}
            \end{figure}
      }
      \only<4>{
            \begin{figure}
                \centering
                \includegraphics[scale=0.2]{images/ub03.jpg}
            \end{figure}
      }
      \only<5>{
            \begin{figure}
                \centering
                \includegraphics[scale=0.2]{images/ub04.jpg}
            \end{figure}
      }
\end{frame}

%%%%%%%%%%%%%%%%%%%%%%%%%%%%%%%%%%%%%%%%%%%%%%%%%%%%%%%%%%%%
% Fachschaftsservices
%%%%%%%%%%%%%%%%%%%%%%%%%%%%%%%%%%%%%%%%%%%%%%%%%%%%%%%%%%%%

\section{Fachschaftsservices}
\begin{frame}{Fachschaftsservices}
\end{frame}

\begin{frame}{Wer will mir hier was erzählen!?}
    \vfill
    \begin{center}
        {\Large Janina Rastetter} \\
        \mail{jrastetter@mathi.uni-heidelberg.de} \\
        \vspace{1em}
        {\Large Christian Heusel } \\
        \mail{chris@mathphys.stura.uni-heidelberg.de} \\
    \end{center}
\end{frame}

\begin{frame}{Vielen Dank fürs Zuhören!}
    \vfill
    \begin{center}
        \Huge Fragen \\[3.5pt]
        \normalsize Serviceangebote Studium
    \end{center}
\end{frame}

\end{document}
