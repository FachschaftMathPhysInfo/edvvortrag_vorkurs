% !TEX root = vortrag.tex
% !TEX encoding = UTF-8 Unicode

\documentclass[t, ngerman, aspectratio=1610, 12pt]{beamer} %compress,

%% Pakete laden...
\usepackage[T1]{fontenc}
\usepackage[utf8]{inputenc}
\usepackage{
    babel,
    bookmark,
    booktabs,
    colortbl,
    graphicx,
    microtype,
    pifont,
    pgfpages,
    tikz,
    url,
    xcolor
}

\usepackage{hyperref}
\usepackage[hyperlink]{qrcode}

%% Design festlegen...
\mode<presentation>{
    %      \useoutertheme[subsection=false]{smoothbars}
    \useinnertheme{rectangles} % rectangles, circles, rounded
    \usecolortheme[RGB={153,0,0}]{structure}
    \usetheme[progressbar=frametitle]{metropolis}
    \beamertemplatenavigationsymbolsempty
    \setbeamerfont{footline}{size=\bf\huge}
    \setbeamerfont{subtitle}{size=\normalsize}
    \setbeamertemplate{note page}[plain]
    \setbeamertemplate{caption}{\raggedright\insertcaption\par} % captions without prefix
    % \logo{\includegraphics[width=3.5cm]{images/MathPhysLogoMathInf.pdf}}
}

%% nützliche Definitionen...
\graphicspath{{images/}}
\newfontfamily\DejaSans{DejaVu Sans}
\newcommand{\mail}[1]{\href{mailto:#1}{\texttt{#1}}}

%% Titelinformationen...
\title{Serviceangebote Studium}
\subtitle{\vspace{-0.35cm}\texttt{EDV so weit das Auge reicht}}
\author[we]{
    \href{mailto:chris@mathphys.stura.uni-heidelberg.de}{Christian Heusel}, Janina Rastetter
}
%  \institute{  \includegraphics[width=5.5cm]{fs-logo-big} }

\date{30. September 2019}

\hypersetup{
    pdfauthor={Christian Heusel, Janina Rastetter},
    pdftitle={Serviceangebote Studium},
    pdfsubject={Begrüßungsvortrag für die angehenden Erstsemester der Fächer Mathematik \& Informatik an der Uni Heidelberg},
    pdfkeywords={EDV, Vorkurs, Fachschaft, MathPhysInfo},
    pdfpagelayout={SinglePage},
}
\newenvironment{rcases}{%
  \left.\renewcommand*\lbrace.%
  \begin{cases}}%
{\end{cases}\right\rbrace}
