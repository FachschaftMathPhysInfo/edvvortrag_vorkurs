% !TEX root = vortrag.tex
% !TEX encoding = UTF-8 Unicode

\documentclass[t, ngerman]{beamer} %compress,

%% Pakete laden...
  \usepackage[T1]{fontenc}
  \usepackage[utf8]{inputenc}
  \usepackage{
      babel,
      bookmark,
      booktabs,
%      blindtext,
      colortbl,
%      eurosym,
      graphicx,
	  hyperref,
%      libertine,
      microtype,
      pifont,
      pgfpages,
      tikz,
%      xspace,
  }


%% Design festlegen...
  \mode<presentation>{
%      \useoutertheme[subsection=false]{smoothbars}
      \useinnertheme{rectangles} % rectangles, circles, rounded
      \usecolortheme[RGB={153,0,0}]{structure}
      \definecolor{unihd}{RGB}{153,0,0}
      \definecolor{dark}{RGB}{115,0,0}
      \definecolor{light}{RGB}{241,229,229}
      \usecolortheme{whale}
		 \usecolortheme{orchid}
%	   \usecolortheme{beaver}
%      \setbeamercovered{transparent}
      \beamertemplatenavigationsymbolsempty
%      \setbeameroption{show notes on second screen}
      \setbeamertemplate{note page}[plain]
      \logo{\includegraphics[width=3.5cm]{fs-logo-small}}

  }

%% nützliche Definitionen...
  \graphicspath{{media/}}

%% Titelinformationen...
  \title[Studienverwaltung$\mu$]{Das Müsli und das LSF\\\small oder: wer wie was wo Stundenplan}
  \author[
	koebi
  ]{
	Jakob Schnell\\{\scriptsize\url{koebi@mathphys.stura.uni-heidelberg.de}}
  }
%  \institute{  \includegraphics[width=5.5cm]{fs-logo-big} }

  \date{\vspace*{-2em}\\ 01. Oktober 2018}

  \hypersetup{
      pdfauthor={Jakob Schnell},
      pdftitle={Müsli-Vortrag},
      pdfsubject={hihi},
      pdfkeywords={1},
      pdfpagelayout={SinglePage},
  }
  


\newenvironment{rcases}{%
  \left.\renewcommand*\lbrace.%
  \begin{cases}}%
{\end{cases}\right\rbrace}
