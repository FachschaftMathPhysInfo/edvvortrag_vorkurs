% !TEX TS-program = pdflatex
% !TEX encoding = UTF-8 Unicode

% !TEX root = present.tex
% !TEX encoding = UTF-8 Unicode

\documentclass[
    t,
    ngerman,
    hyperref={
        colorlinks=false
    }
    ]{beamer} %compress,

%% Pakete laden...
  \usepackage[T1]{fontenc}
  \usepackage[utf8]{inputenc}
  \usepackage{
      babel,
      bookmark,
      booktabs,
%      blindtext,
      colortbl,
%      eurosym,
      graphicx,
      hyperref,
%      libertine,
      microtype,
      pifont,
      pgfpages,
      tikz,
      wrapfig,
      qrcode,
%      xspace,
  }


%% Design festlegen...
  \mode<presentation>{
%      \useoutertheme[subsection=false]{smoothbars}
      \useinnertheme{rectangles} % rectangles, circles, rounded
      \usecolortheme[RGB={153,0,0}]{structure}
      \definecolor{unihd}{RGB}{153,0,0}
      \definecolor{dark}{RGB}{115,0,0}
      \definecolor{light}{RGB}{241,229,229}
      \usecolortheme{whale}
         \usecolortheme{orchid}
%      \usecolortheme{beaver}
%      \setbeamercovered{transparent}
      \beamertemplatenavigationsymbolsempty
%      \setbeameroption{show notes on second screen}
      \setbeamertemplate{note page}[plain]
      \logo{\includegraphics[width=3.5cm]{./media/fs-logo-small.pdf}}

  }

%% nützliche Definitionen...
  \graphicspath{{media/}}

%% Titelinformationen...
  \title[EDV]{Serviceangebote Studium\\(VPN, eduroam, URZ, drucken, UB)}
  \author[
    Patrick und Chris
  ]{
    Patrick Dammann\\{\scriptsize\url{patrick@mathphys.stura.uni-heidelberg.de}}\\
    Christian Heusel\\{\scriptsize\url{chris@mathphys.stura.uni-heidelberg.de}}
}
%  \institute{  \includegraphics[width=5.5cm]{fs-logo-big} }

  \date[02.10.2017]{\vspace*{-2em}\\ 02. Oktober 2018}

  \hypersetup{
      pdfauthor={Patrick Damann, Christian Heusel},
      pdftitle={Serviceangebote Studium},
      pdfsubject={Serviceangebote Studium},
      pdfkeywords={1},
      pdfpagelayout={SinglePage},
  }



\newenvironment{rcases}{%
  \left.\renewcommand*\lbrace.%
  \begin{cases}}%
{\end{cases}\right\rbrace}


\begin{document}
% !TEX root = present.tex
% !TEX encoding = UTF-8 Unicode

\begin{frame}[plain]
       \bookmark[level=0, page=1]{Titel}
       \titlepage
        \tikz[,overlay]
     \node at
        (current page.south)
        {\includegraphics[width=5cm]{MathPhysLogoMathInf.pdf}};
\end{frame}

\begin{frame}{Übersicht}
  \bookmark[level=0, page=2]{Inhalt}
  \tableofcontents
    \note{\tableofcontents}
    Folien: \url{http://mathphys.info/vorkurs/plan}
\end{frame}


\section{Benutzerkennung freischalten}
\begin{frame}{Benutzerkennung freischalten}
    \begin{itemize}
        \item Benutzerkennung wird an den meisten Rechnern und Computersystemen Uni-weit verwendet
        \item muss freigeschaltet werden
        \item auf \textcolor{blue}{\href{https://public.urz.uni-heidelberg.de/cgi-bin/wi/freisch.php?funk=2}{www.urz.uni-heidelberg.de $\rightarrow$ Kennung freischalten $\rightarrow$ mit Uni-ID und Aktivierungscode}}
            gehen und Uni-ID (auf Studiausweis), Aktivierungscode (mit
            Immatrikulationsunterlagen bekommen) und ein neues
            Kennwort eingeben
        \item Passwort kann unter \\
        {\url{www.urz.uni-heidelberg.de/de/passwort-erneuern}}
        geändert werden
        \item Alles zu finden unter\\{\url{https://urz.uni-heidelberg.de}} $\rightarrow$ Service-Katalog
    \end{itemize}
\end{frame}

\section{Computer-Pools}
\begin{frame}{Computer-Pools}
\fontsize{8}{9.2}\selectfont
    Mehrere Computer-Pools, die für Mathe und Info zugänglich sind:
    \begin{itemize}
        \item in der \textbf{Bibliothek im Mathematikon} (Windows)\\
            Öffnungszeiten: Montag bis Freitag: 9-21 Uhr, Samstag: 9-15 Uhr\\
            {\url{https://www.mathinf.uni-heidelberg.de/rechnerpools.html}}
        \item am \textbf{Universitätsrechenzentrum} (URZ, Windows, Mac):\\
            Öffnungszeiten: Montag bis Freitag: 8:00-23:45 Uhr, Samstag: 10:00-18:45 Uhr\\
        \item am \textbf{Kirchhoff-Institut für Physik (INF 227)} (Linux)\\
            Öffnungszeiten: Montag bis Freitag: 8-19 Uhr\\
            {\url{https://www.kip.uni-heidelberg.de/cip/cip.html}}
        \item am \textbf{Physikalisches Institut (INF 226)} (Linux)\\
            Öffnungszeiten: Montag bis Freitag: 8-19 Uhr\\
            {\url{https://www.physi.uni-heidelberg.de/Einrichtungen/CIP/}}
        \item am \textbf{Info-Café der Mensa (INF 304)} (Windows)\\
            Öffnungszeiten: Montag bis Donnerstag: 9-17 Uhr, Freitag 9-15 Uhr\\
            {\url{http://stw.uni-heidelberg.de/de/oeffnungszeiten}}
    \end{itemize}
\end{frame}

\section{Netzzugang}
\begin{frame}{Netzzugang}
    \begin{itemize}
        \item Benötigt VPN $\Rightarrow$ mehr dazu später
        \item Netzwerkdosen im URZ (1. Stock), KIP, Mathematikon …
        \item WLAN Zugang auf dem ganzen Campus\pause
            \begin{itemize}
                \item eduroam
                \item UNI-HEIDELBERG
                \item UNI-WEBACCESS
                \item Heidelberg4you
            \end{itemize}\pause
        \item VPN
    \end{itemize}
\end{frame}

\subsection{WLAN}
\begin{frame}{Netzzugang - WLAN - eduroam - Allgemein}
    \begin{itemize}
        \item Netzabdeckung Uniweit ziemlich gut (außer in Mensen)
        \item funktioniert auch an vielen anderen Unis deutschland- und weltweit
            {\url{https://www.eduroam.org/where}}
        \item Manche Dienste/Programme funktionieren im eduroam in Heidelberg nicht
    \end{itemize}
\end{frame}

\begin{frame}{Netzzugang - WLAN - eduroam - cat.eduroam.de}
    \begin{itemize}
        \item \url{https://cat.eduroam.de/}
        \item Auch für abstruse Plattformen!
    \end{itemize}
    \begin{center}
    \qrcode[height=130pt]{https://cat.eduroam.de/}
    \end{center}
\end{frame}

\begin{frame}{Netzzugang - WLAN - eduroam - manuell}
    \begin{itemize}
        \item \textbf{Zugangsdaten}:\\
        \begin{description}
            \item[SSID:] eduroam\\
            \item[User:] \textit{UNI-ID}@uni-heidelberg.de\\
            \item[Password:] URZ Kennwort\\\pause
        \end{description}
        \item weitere Angaben, die manchmal hilfreich sind:\\
        \begin{description}
            \item[Verschlüssellungsmethode:] WPA2 (Enterprise)\\
            \item[EAP-Methode:] TTLS\\
            \item[Phase2-Authentifizierung:] PAP\\
            \item[CA-Zertifikat:] Dt. Telekom2 (vorher herunterladen)\\
            \item[DNS:] radius.urz.uni-heidelberg.de\\
            \item[DNS2:] radius2.urz.uni-heidelberg.de\\
            \item[Anonymous:] anonymous@uni-heidelberg.de\pause
        \end{description}
    \end{itemize}
\end{frame}

\begin{frame}{Netzzugang - WLAN - UNI-HEIDELBERG}
    \begin{itemize}
        \item es funktionieren alle Dienste
        \item man braucht eine VPN-Verbindung um Internet zu erhalten
        \item mit „UNI-HEIDELBERG“ verbinden, dann VPN-Client starten
    \end{itemize}
\end{frame}

\begin{frame}{Netzzugang - WLAN - UNI-WEBACCESS}
    \begin{itemize}
        \item \textit{sehr} simpel
        \item nicht sicher
        \item extrem langsam \& instabil
        \item mit „UNI-WEBACCESS“ verbinden, im Browser UNI-ID und Passwort eingeben
        \item nicht für Dauereinsatz
    \end{itemize}
\end{frame}

\subsection{Kabelgebunden}
\begin{frame}{Netzzugang - Kabelgebunden}
    \begin{itemize}
        \item verschiedene Netzwerke an verschiedene Netzwerkdosen
        \item Laptop-LAN = UNI-HEIDELBERG WLAN
        \item Laptop-LAN in den meisten Seminarräumen
        \item $\exists$ Dosen $\colon \nexists$ Netz
        \item Instituts-Netze
    \end{itemize}
\end{frame}

\subsection{VPN}
\begin{frame}{Netzzugang - VPN}
    \begin{itemize}
        \item nötig um manche Uni-Seiten von außen zu erreichen \\ $\Rightarrow$ (z.B. PW Reset)
        \item oder um sich mit „UNI-HEIDELBERG“, bzw. dem Laptop-LAN zu verbinden\pause
        \item Zugang zu vielen wissenschaftlichen Forschungsportalen/Papern
        \item Offiziellen VPN-Client bekommt man von {\url{https://vpn-ac.urz.uni-heidelberg.de}}\\
            Dieser muss nur installiert werden und als Server dann
            \textbf{vpn-ac.urz.uni-heidelberg.de} eingetragen werden.
        \item Funktioniert mit Linux, Windows, Mac OS X, Android und iOS gleichermaßen.
    \end{itemize}
\end{frame}

\begin{frame}{Netzzugang - VPN}
    \begin{itemize}
        \item Es können auch native Clients verwendet werden die oft angenehmer zu nutzen sind.
        \item Unter Linux nutzt man openconnect, was in den meisten Distributionen verfügbar ist.
        \item Für Android und iOS gibt es ebenfalls openconnect, welches äquivalent zum Any-Connect client zu nutzen ist.\pause
        \item weitere Infos: \url{https://www.urz.uni-heidelberg.de/de/vpn}
    \end{itemize}
\end{frame}

\section{Drucken}
\begin{frame}{Drucken}
    \begin{itemize}
        \item dezentrales Drucksystem\pause
        \item \textbf{Anleitung:}
            \begin{itemize}
                \item auf \url{http://drucker.uni-hd.de} gehen und sich
                    einloggen
                \item Dokument (PDF, OpenOffice, Microsoft Office, JPG, GIF, PNG, TIF, BMP, TXT) hochladen
                \item an einen Drucker/Kopierer gehen \\
                    (Mathematikon, Theoretikum, (INF 306) und im URZ)
                \item CampusCard reinstecken, anmelden, Druck-Symbol auswählen
                    und dem Menü folgen
                \item für doppelseitigen Druck auf entsprechenden Druckauftrag klicken und einstellen.
            \end{itemize}\pause
        \item weitere Infos und ausführliche Anleitung zum Drucken (und
            Scannen) auf den Webseiten des URZ und über den Druckern selbst.
        \item \textbf{Tipp:} 20 Seiten pro Woche können auch im KIP-CIP-Pool gedruckt werden
    \end{itemize}
\end{frame}

\section{Mail}
\begin{frame}{Mail}
    \begin{itemize}
        \item erreichbar unter \url{https://sogo.urz.uni-heidelberg.de}
        \item \textbf{unbedingt} abrufen! Darüber laufen wichtige Studiumsrelevante Dinge.
        \item Weiterleitung an externe Adresse möglich
        \item Eigenen Mailclient einrichten: \\ \url{https://www.urz.uni-heidelberg.de/de/e-mail-und-groupware}
        \item Kurz Infos: \\ \url{https://www.urz.uni-heidelberg.de/de/uebersicht-e-mail-server}
    \end{itemize}
\end{frame}

\section{Software und Anleitungen}
\begin{frame}{Software und Anleitungen}
    \begin{itemize}
        \item Anleitungen und Hinweise zu allen Services des URZ unter {\url{https://urz.uni-heidelberg.de}}
        \item Software-Lizenzen (auch für Privatrechner) gibt es unter
            {\url{www.urz.uni-heidelberg.de/de/lizenzmanagement}}
        \item über MS Imagine Premium erhaltet ihr (unter anderem) eine Windows Lizenz umsonst\\
            {\small\url{www.urz.uni-heidelberg.de/de/microsoft-imagine-premium}}
    \end{itemize}
\end{frame}

\section{Universitätsbibliothek Services}
\begin{frame}{Universitätsbibliothek Services}
    \begin{itemize}\pause
        \item HEIDI {(\url{http://heidi.uni-hd.de})} ist der Bücherkatalog der UBs\pause
        \item eBooks {(\url{http://ebooks.uni-hd.de})} gibt es auch über HEIDI\pause
        \item Springer Link angebot über HEIDI
        \item Ausleihen geht in INF 368 im 3. Stock\pause
        \item Buchscanner im Erdgeschoss, INF 368\pause
    \end{itemize}
\end{frame}

\section{Verschiedenes}
\begin{frame}{uni-heidelberg.de vs uni-hd.de}
    \begin{itemize}
        \item uni-hd.de war ursprünglich für verkürzte Links gedacht
        \item viele Webseiten waren nur über diese Adresse erreichbar
        \item Existiert eigentlich nicht mehr
        \item es läuft jetzt alles über uni-heidelberg.de
    \end{itemize}
\end{frame}

\section{Fachschaft}
\begin{frame}{Studentische offene Sprechstunde (SoS)}
    \begin{itemize}
        \item 
        \item viele Webseiten waren nur über diese Adresse erreichbar
        \item Existiert eigentlich nicht mehr
        \item es läuft jetzt alles über uni-heidelberg.de
    \end{itemize}
\end{frame}

\begin{frame}{ENDE}
    \center
    Folien alle auf \url{http://mathphys.info/vorkurs/plan}
\end{frame}
\end{document}
