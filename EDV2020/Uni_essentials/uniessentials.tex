%pdf-latex
% !TeX program = xelatex
%header nur vom MÜSLI-Vortrag geklaut
% !TEX root = present.tex
% !TEX encoding = UTF-8 Unicode

\documentclass[
    t,
    ngerman,
    hyperref={
        colorlinks=false
    }
    ]{beamer} %compress,

%% Pakete laden...
  \usepackage[T1]{fontenc}
  \usepackage[utf8]{inputenc}
  \usepackage{
      babel,
      bookmark,
      booktabs,
%      blindtext,
      colortbl,
%      eurosym,
      graphicx,
      hyperref,
%      libertine,
      microtype,
      pifont,
      pgfpages,
      tikz,
      wrapfig,
      qrcode,
%      xspace,
  }


%% Design festlegen...
  \mode<presentation>{
%      \useoutertheme[subsection=false]{smoothbars}
      \useinnertheme{rectangles} % rectangles, circles, rounded
      \usecolortheme[RGB={153,0,0}]{structure}
      \definecolor{unihd}{RGB}{153,0,0}
      \definecolor{dark}{RGB}{115,0,0}
      \definecolor{light}{RGB}{241,229,229}
      \usecolortheme{whale}
         \usecolortheme{orchid}
%      \usecolortheme{beaver}
%      \setbeamercovered{transparent}
      \beamertemplatenavigationsymbolsempty
%      \setbeameroption{show notes on second screen}
      \setbeamertemplate{note page}[plain]
      \logo{\includegraphics[width=3.5cm]{./media/fs-logo-small.pdf}}

  }

%% nützliche Definitionen...
  \graphicspath{{media/}}

%% Titelinformationen...
  \title[EDV]{Serviceangebote Studium\\(VPN, eduroam, URZ, drucken, UB)}
  \author[
    Patrick und Chris
  ]{
    Patrick Dammann\\{\scriptsize\url{patrick@mathphys.stura.uni-heidelberg.de}}\\
    Christian Heusel\\{\scriptsize\url{chris@mathphys.stura.uni-heidelberg.de}}
}
%  \institute{  \includegraphics[width=5.5cm]{fs-logo-big} }

  \date[02.10.2017]{\vspace*{-2em}\\ 02. Oktober 2018}

  \hypersetup{
      pdfauthor={Patrick Damann, Christian Heusel},
      pdftitle={Serviceangebote Studium},
      pdfsubject={Serviceangebote Studium},
      pdfkeywords={1},
      pdfpagelayout={SinglePage},
  }



\newenvironment{rcases}{%
  \left.\renewcommand*\lbrace.%
  \begin{cases}}%
{\end{cases}\right\rbrace}


\begin{document}

\begin{frame}[fragile]
    \maketitle{}
\end{frame}

\begin{frame}{Inhalt des Vortrags}
    \begin{minipage}[t]{0.515\textwidth}
        \tableofcontents[hideallsubsections, sections={1-5}]
    \end{minipage}
    \begin{minipage}[t]{0.475\textwidth}
        \tableofcontents[hideallsubsections, sections={6-11}]
    \end{minipage}
\end{frame}

\begin{frame}{Wer will mir hier was erzählen!?}
    \vfill
    \begin{center}
        {\Large Johanna Riedel} \\
        \mail{johanna@mathphys.stura.uni-heidelberg.de} \\
        \vspace{1em}
        {\Large Christian Heusel } \\
        \mail{chris@mathphys.stura.uni-heidelberg.de} \\
    \end{center}
\end{frame}

\begin{frame}{Präambel}
    \Large
    \begin{center}
        \only<1->{
           {\huge \DejaSans{} 😇} Es wird die Folien zum Download geben! {\huge \DejaSans{} 😇}\\[0.7em]
        }
        \only<1>{\includegraphics[scale=0.6]{images/xkcd_1.png}}
        \only<2->{
        {\huge \DejaSans{}😏 } Der Vortrag wird auch hochgeladen! {\huge \DejaSans{}😏 } }
    
        \only<3->{
           {\huge \DejaSans{} 😺} Stellt Fragen, im Währenden oder am Ende! {\huge \DejaSans{} 😺}\\[0.7em]
        }
        \only<4->{
            \includegraphics[]{images/lets_go.png}
        }
    \end{center}
\end{frame}

%%%%%%%%%%%%%%%%%%%%%%%%%%%%%%%%%%%%%%%%%%%%%%%%%%%%%%%%%%%%
% Emails
%%%%%%%%%%%%%%%%%%%%%%%%%%%%%%%%%%%%%%%%%%%%%%%%%%%%%%%%%%%%

\section{Uni-Mail-Adresse}
\begin{frame}{Uni-Mail-Adresse}
	\begin{itemize}
		\item<+-> jeder Studierende bekommt vom URZ eine Mail-Adresse \\
		\texttt{<uni-id>@stud.uni-heidelberg.de} \\
		\item<+->diese Adresse ist abrufbar via Webinterface: \\
		\url{https://sogo.urz.uni-heidelberg.de} \\
		\item<+-> oder mit einem Mailclient (z.B. Thunderbird): \\
		\only<3>{
			\vspace{0.5em}
			\begin{tabular}{lll}
				Aufgabe & Adresse & Port \\ \toprule
				Posteingangsserver & \texttt{imap.urz.uni-heidelberg.de} & \texttt{993} \\
				Postausgangsserver & \texttt{mail.urz.uni-heidelberg.de} & \texttt{587} \\
				\href{https://www.urz.uni-heidelberg.de/de/uebersicht-e-mail-server}{\scriptsize Link zur Quelle} & & \\
			\end{tabular} \\
		}
		\item<+-> \emph{WICHTIG:} diese Mails solltet ihr lesen
		\item<+-> Man kann die Mails auf ein privates Konto weiterleiten \\
		SoGO: Preferences > Forward > Forward incoming messages (+ copy)
	\end{itemize}
\end{frame}

\begin{frame}{Mail Alias}
	\begin{itemize}
		\item<1-2> es ist möglich eine zusätzliche Mail-Adresse zu erhalten \\
		\texttt{<name>@stud.uni-heidelberg.de}
		\item<2> man muss ein Formular ausfüllen \\
		\href{https://it-service.uni-heidelberg.de/anfrage/mail_alias_beantragen}{\scriptsize Link zum Formular} 
		\end{itemize}
		\only<3>{
			\begin{center}
				\vspace{-7em}
				\includegraphics[width=1.0\textwidth]{Alias.png}
			\end{center}
		}
\end{frame}

%%%%%%%%%%%%%%%%%%%%%%%%%%%%%%%%%%%%%%%%%%%%%%%%%%%%%%%%%%%%
% Webkonferenzen
%%%%%%%%%%%%%%%%%%%%%%%%%%%%%%%%%%%%%%%%%%%%%%%%%%%%%%%%%%%%

\section{Webkonferenzen}
\begin{frame}{Übersicht: Webkonferenzen}
	\Large
	\begin{itemize}
		\item Etikette
		\item Verwendete Tools
		\item Weitere sinnvolle Tools
	\end{itemize}
    
\end{frame}

\subsection{Etikette}
\begin{frame}{Etikette}
Wie verhalte ich mich richtig in Onlinevorlesungen und Tutorien?
	\begin{itemize}
		\item rechtzeitig beitreten
		\item eigenes Video ausschalten, um Bandbreite zu sparen
		\item Headset mit Mikrofon für gute Tonqualität verwenden
		\item Mikrofon stumm schalten
		\item Möglichkeiten der Plattformen nutzen \\
		Hand heben, Chats, Abstimmungen, etc.
		\item Verständnis für die Vortragenden haben
	\end{itemize}
\end{frame}

\subsection{Verwendete Tools}
\begin{frame}{\includegraphics[scale=0.092]{images/zoom.jpg} Zoom}
	\url{https://zoom.us/}
	\begin{itemize}
		\item genutzt für Onlinevorlesungen (LA, Ana, ITE, etc.)
		\item die Fragestunde der LA und Ana findet hier statt
		\item es gibt einen Client, den man installieren sollte
		\item man benötigt keinen Account
	\end{itemize}
	
\end{frame}

\begin{frame}{\includegraphics[scale=0.42]{images/heiConf.png} heiCONF}
	\url{https://heiconf.uni-heidelberg.de/}
	\begin{itemize}
		\item genutzt für Onlinevorlesungen (Programmierkurs, $\dots$)
		\item hier finden die meisten Tutorien statt
		\item wird im Browser verwendet
		\item keine Anmeldung notwendig
		\item verlangt Name und Mail-Adresse beim Beitreten
		\item besitzt Meldefunktion
		\item nur Mitarbeiter der Uni können einen Account erstellen
	\end{itemize}
	
\end{frame}

\subsection{Verwendete Tools}
\begin{frame}{\includegraphics[scale=0.092]{images/webex.png} Cisco WebEx}
	\url{https://www.webex.com/de/index.html}
	\begin{itemize}
		\item manche Vorlesungen und Tutorien verwenden WebEx
		\item es gibt einen Client, den man bei Bedarf installieren kann
		\item ist auch im Browser angenehm verwendbar
		\item man benötigt keinen Account
	\end{itemize}
	
\end{frame}

\subsection{Discord}
\begin{frame}{\includegraphics[scale=0.092]{images/discord.png} Discord}
	\url{https://discord.com/}
	\begin{itemize}
		\item Kommunikationsplattform für Help-Desk und Zettelgruppen
		\item wird im Vorkurs verwendet
		\item praktisch um zusammen zu lernen oder Zeit zu verbringen
	\end{itemize}
\end{frame}

%%%%%%%%%%%%%%%%%%%%%%%%%%%%%%%%%%%%%%%%%%%%%%%%%%%%%%%%%%%%
% Vorlesungswebseiten
%%%%%%%%%%%%%%%%%%%%%%%%%%%%%%%%%%%%%%%%%%%%%%%%%%%%%%%%%%%%

\section{Vorlesungswebseiten}
\begin{frame}{Vorlesungswebseiten}
    \begin{itemize}
    	\item Lineare Algebra I \& Analysis I \\
    	\url{https://www.mathi.uni-heidelberg.de/~diffgeo/LA_Ana_WS2021}
    	\item Fun Facts
aus der
Analysis und Linearen Algebra \\
    	\url{https://www.mathi.uni-heidelberg.de/~aschilling/teaching/funfacts-WS20}
    	\item Einführung in die Praktische Informatik\\
    	\url{https://www.math.uni-heidelberg.de/logic/ws20/ipi_ws20.html}
    	\item Einführung in die Technische Informatik \\
    	\url{https://ra.ziti.uni-heidelberg.de/cag/teaching}
    	\item Programmierkurs \\
    	\url{https://conan.iwr.uni-heidelberg.de/teaching/ipk_ws2020/}
    \end{itemize}
\end{frame}

%%%%%%%%%%%%%%%%%%%%%%%%%%%%%%%%%%%%%%%%%%%%%%%%%%%%%%%%%%%%
% Wichtige Tools
%%%%%%%%%%%%%%%%%%%%%%%%%%%%%%%%%%%%%%%%%%%%%%%%%%%%%%%%%%%%

\section{Wichtige Tools im Studium}
\begin{frame}{Übersicht: Wichtige Tools im Studium}
	\Large
    \begin{itemize}
        \item Mampf
        \item Moodle
        \item Müsli
    \end{itemize}
\end{frame}


%%%%%%%%%%%%%%%%%%%%%%%%%%%%%%%%%%%%%%%%%%%%%%%%%%%%%%%%%%%%
% MaMpf
%%%%%%%%%%%%%%%%%%%%%%%%%%%%%%%%%%%%%%%%%%%%%%%%%%%%%%%%%%%%

\subsection{MaMpf}
\begin{frame}{\includegraphics[scale=0.072]{images/mampf.png} MaMpf -- Mathematische Medienplattform}
	
	Link: \url{https://mampf.mathi.uni-heidelberg.de/}
	
	\begin{itemize}
		\item Speziell fürs Mathelernen konzipierte E-Learning-Plattform
		\item Wird hier am Mathematischen Institut betrieben und entwickelt
		\item Viele Lernangebote (Vorlesungsvideos, Beispielsvideos, Quizzes, ...),\\
		die getaggt und untereinander vernetzt sind
		\item Werdet ihr benutzen, wenn ihr die LA1 und Ana1 hört
		%\item Werdet ihr auch im Vorkurs brauchen (?)
		\item Mehr zu MaMpf: \url{https://mampf.blog/}
	\end{itemize}
	
\end{frame}

\begin{frame}{\includegraphics[scale=0.072]{images/mampf.png} MaMpf -- Registrierung}
		Registriert euch einfach mit einer x-beliebigen E-Mail-Adresse.\\
		\begin{figure}
			\centering
			\includegraphics[scale=0.3]{images/mampf01.jpg}
		\end{figure}
\end{frame}

\begin{frame}{\includegraphics[scale=0.072]{images/mampf.png} MaMpf -- Tour}
	\vspace{6em}
	\begin{center}
		\Large Eine Tour findet ihr im hochgeladenen Vortragsvideo!
	\end{center}
\end{frame}

%%%%%%%%%%%%%%%%%%%%%%%%%%%%%%%%%%%%%%%%%%%%%%%%%%%%%%%%%%%%
% Moodle
%%%%%%%%%%%%%%%%%%%%%%%%%%%%%%%%%%%%%%%%%%%%%%%%%%%%%%%%%%%%
\subsection{Moodle}
\begin{frame}{Moodle -- uniweite E-Learningplattform}
	
	Link: \url{https://moodle.uni-heidelberg.de/login/index.php}
	
	\begin{itemize}
		\item Anmeldung über Uni-ID
		\item{Kurse, in die ihr euch mit einem Einschreibeschlüssel eintragen könnt; Dozent*innen geben den Einschreibeschlüssel i.d.R. in der ersten Sitzung bekannt}
		\item{Materialien zu Veranstaltungen wie Übungsblätter oder Präsentationen}
		\item{Abgaben}
	\end{itemize}
	
\end{frame}

\begin{frame}{Moodle -- uniweite E-Learningplattform}
	\only<1>
		\vspace{6em}
		\begin{center}
		\Large Eine Tour findet ihr im hochgeladenen Vortragsvideo!
		\end{center}
	}
\end{frame}

\begin{frame}{Was muss ich bei Zettelabgaben beachten?}
	\begin{itemize}
		\item Zettel werden voraussichtlich über MaMpf (Mathe) und Moodle (Info) abgegeben
		\item gebt die Zettel pünktlich ab $\rightarrow$ sonst eventuell keine Punkte
		\item die genauen Abgaberegelungen werden euch rechtzeitig in eurer Vorlesung mitgeteilt
		\item Scans oder gute Fotos eurer Antworten reichen i.d.R. aus
		\item Code wird als Sourcefile abgegeben
	\end{itemize}
\end{frame}


%%%%%%%%%%%%%%%%%%%%%%%%%%%%%%%%%%%%%%%%%%%%%%%%%%%%%%%%%%%%
% MÜSLI
%%%%%%%%%%%%%%%%%%%%%%%%%%%%%%%%%%%%%%%%%%%%%%%%%%%%%%%%%%%%

\subsection{MÜSLI}
\begin{frame}{MÜSLI -- \normalsize Mathematisches Übungsgruppen- und Scheinlisten-Interface}

    \large \url{https://muesli.mathi.uni-heidelberg.de/}

    \begin{minipage}[t]{0.7\textwidth}

    \begin{itemize}
        \item Eintragung in Übungsgruppen
        \item Einsehen von Zettelpunkten
        \item E-Mail-Adressen der Tutor*innen
        \item Klausuranmeldung
        \item Noten
    \end{itemize}
    \end{minipage}
    \begin{minipage}[t]{0.28\textwidth}
        \vspace*{0em}
        \begin{center}
            \qrcode[height=0.9\textwidth]{https://muesli.mathi.uni-heidelberg.de/}
        \end{center}
    \end{minipage}
\end{frame}

\begin{frame}{MÜSLI -- \normalsize Mathematisches Übungsgruppen- und Scheinlisten-Interface}
    \Large {\LARGE \DejaSans{} ⚠} Achtung {\LARGE \DejaSans{}⚠} \\
    \normalsize
    \begin{enumerate}
        \item{Ihr müsst euch mit eurer Uni-Mail-Adresse registrieren.}
        \item{In Mathe/Info: Anmeldung für Übungsgruppe = Anmeldung für Vorlesung}
        \item{In anderen Fächern gibt es andere Anmeldemodalitäten. Wenn ihr Kurse in einem anderen Fach belegen wollt/müsst, informiert euch frühzeitig(!) darüber, wie ihr euch in diesem Fach zu Veranstaltungen anmelden könnt.}
    \end{enumerate}
\end{frame}

\begin{frame}{MÜSLI -- \normalsize Mathematisches Übungsgruppen- und Scheinlisten-Interface}
	 \Large {\LARGE \DejaSans{} ⚠} Achtung {\LARGE \DejaSans{}⚠} \\
	\normalsize
\begin{itemize}
	\item Eure Anmeldungen in LA / Ana sind noch nicht fest
	\item es gibt ein neues Präferenzsystem zur Übungsgruppenzuteilung
	\item sprecht euch mit euren Freunden ab und gebt die gleichen Präferenzen an
	\item dann: höhere Chance auf die gleichen Übungsgruppen
	\item \textbf{aber:} keine Garantie!
\end{itemize}
\end{frame}

%%%%%%%%%%%%%%%%%%%%%%%%%%%%%%%%%%%%%%%%%%%%%%%%%%%%%%%%%%%%
% Wie soll ich mir diese ganzen Links merken?
%%%%%%%%%%%%%%%%%%%%%%%%%%%%%%%%%%%%%%%%%%%%%%%%%%%%%%%%%%%%
\begin{frame}{Wie soll ich mir diese ganzen Links merken?}
	\begin{itemize}
		\item das musst du auch nicht
		\item lass das doch einfach deinen Rechner erledigen
	\end{itemize}
	\textbf{Tipps:}
	\begin{itemize}
		\item speichere Webseiten, die du oft benötigst als Lesezeichen in deinem Browser
		\item wenn du noch etwas dazu speichern musst (z.B. einen Zugangscode), leg dir eine Datei an, in der du alle wichtigen Links speicherst
	\end{itemize}
\end{frame}

%%%%%%%%%%%%%%%%%%%%%%%%%%%%%%%%%%%%%%%%%%%%%%%%%%%%%%%%%%%%
% HelpDesk Vorstellung
%%%%%%%%%%%%%%%%%%%%%%%%%%%%%%%%%%%%%%%%%%%%%%%%%%%%%%%%%%%%
\section{HelpDesk Vorstellung}
\begin{frame}{HelpDesk -- Hürden abbauen}
    \only<1-3>{%
        \large \textbf{Wofür sind wir da?}
        \normalsize
        \begin{itemize}
            \item<1-> Ich habe so grundlegende inhaltliche Fragen, dass ich sie im Tutorium gar nicht stellen kann oder will
                $\Rightarrow$ \textbf{fachlicher HelpDesk}
            \item<2-> Ich brauche Hilfe bei der Organisation meines Studiums, will aber nicht direkt mit einem Dozenten reden
                $\Rightarrow$ \textbf{Drumherum-HelpDesk}
        \end{itemize}
            \only<3->{% TODO Hier aktuelle Gegebenheiten des HelpDesk einfügen
                Seminarraum Statistik, 13-16 Uhr (außer Mittwoch) \\
                oder: per Mail an \mail{helpdesk@mathinf.uni-heidelberg.de}
            }
        }
    \only<4>{%
        \vspace{-1em}
        \includegraphics[width=1\textwidth]{help_desk.png}
    }
\end{frame}

%%%%%%%%%%%%%%%%%%%%%%%%%%%%%%%%%%%%%%%%%%%%%%%%%%%%%%%%%%%%
% Fachschaftsservices
%%%%%%%%%%%%%%%%%%%%%%%%%%%%%%%%%%%%%%%%%%%%%%%%%%%%%%%%%%%%

\section{Fachschaftsservices}
\begin{frame}{Fachschaftsservices}
	\begin{itemize}[<+->]
		\item \textbf{Unsere Webseite:} \href{mathphys.info}{\texttt{mathphys.info}} \\
		Veranstaltungshinweise, Informationen über Gremienmitglieder
		\item \textbf{Der Kummerkasten:} \href{https://kummerkasten.mathphys.info}{\texttt{kummerkasten.mathphys.info}} \\
		anonyme Rückmeldung an Dozenten der Vorlesungen
		\item \textbf{Klausurenordner:} \\
		Im Fachschaftsraum (INF 205 // 01.301) gibt es alte Klausuren
		\item \textbf{Mailinglisten:}
		\begin{itemize}
			\item[--] \mail{fachschaft@mathphys.stura.uni-heidelberg.de} \\
			\item[--] \mail{sos@mathphys.stura.uni-heidelberg.de} \\
		\end{itemize}
	\end{itemize}
\end{frame}

%%%%%%%%%%%%%%%%%%%%%%%%%%%%%%%%%%%%%%%%%%%%%%%%%%%%%%%%%%%%
% Software/Anleitungen
%%%%%%%%%%%%%%%%%%%%%%%%%%%%%%%%%%%%%%%%%%%%%%%%%%%%%%%%%%%%

\section{Software \& Anleitungen}
\begin{frame}{Software/Anleitungen}
    Über das URZ kann man vergünstigt/kostenlos Software erwerben:
    \begin{itemize}
        \item Antivirensoftware \pause
        \item Microsoft Office 365 Pro Plus (3,99 € / Jahr) \pause
        \item Developement Software \pause
        \item Windows Lizenzen \pause
    \end{itemize}
    \url{https://www.urz.uni-heidelberg.de/de/lizenzmanagement}
\end{frame}

%%%%%%%%%%%%%%%%%%%%%%%%%%%%%%%%%%%%%%%%%%%%%%%%%%%%%%%%%%%%
%Handlungsaufforderung
%%%%%%%%%%%%%%%%%%%%%%%%%%%%%%%%%%%%%%%%%%%%%%%%%%%%%%%%%%%%

\section{Was sollte ich nochmal alles machen?}
\begin{frame}{Was sollte ich nochmal alles machen?}
	\textbf{Accounts anlegen}
	\normalsize
	\begin{itemize}
		\item bei MaMpf
		\item bei Müsli
	\end{itemize}
	\textbf{Clients installieren}
	\normalsize
	\begin{itemize}
		\item Zoom
		\item WebEx
		\item Discord
	\end{itemize}
	\textbf{Und sonst?}
	\normalsize
	\begin{itemize}
		\item eventuell Mail-Alias beantragen
	\end{itemize}
\end{frame}

\begin{frame}{Vielen Dank fürs Zuhören!}
	\vfill
	\begin{center}
		\Huge Fragen \\[3.5pt]
		\normalsize Uni-Essentials
	\end{center}
\end{frame}


\end{document}
